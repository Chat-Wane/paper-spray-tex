
\begin{abstract}
  The introduction of WebRTC has opened a new playground for
  large-scale distributed applications consisting of enormous numbers of
  directly-communicating web browsers. In this context, gossip-based
  peer-sampling protocols appear as a particularly promising tool
  thanks to their inherent ability to build overlay networks that can
  cope with network dynamics. However, the dynamic nature of
  browser-to-browser communication combined with the connection
  establishment procedures that characterize WebRTC make current
  peer-sampling solutions inefficient or simply unreliable.  In this
  paper, we address the limitations of current peer-sampling
  approaches by introducing \SPRAY, a novel peer-sampling protocol
  designed to meet the constraints introduced by WebRTC. Unlike most
  recent peer-sampling approaches, \SPRAY has the ability to adapt
  its operation to networks that can grow or shrink very
  rapidly. Moreover, by using only neighbor-to-neighbor interactions,
  it limits the impact of the three-way connection establishment
  process that characterizes WebRTC. Our experiments demonstrate the
  ability of  \SPRAY to adapt to dynamic networks and highlight its
  efficiency improvements with respect to existing protocols. 
\end{abstract}

\category{C.2.1}{Computer Systems Organization}{Computer-Communication Networks}[Network Architecture and Design]


\keywords{Large scale distributed applications, random peer sampling,
  WebRTC}

%%% Local Variables:
%%% mode: latex
%%% TeX-master: "../paper"
%%% End:
