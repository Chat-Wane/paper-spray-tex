
\begin{abstract}
  The introduction of WebRTC has opened a new playground for large-scale
  distributed applications consisting of large numbers of directly-communicating
  web browsers. In this context, gossip-based peer-sampling protocols appear as
  a particularly promising tool thanks to their inherent ability to build
  overlay networks that can cope with network dynamics. However, \TODO{the
    dynamicity of web applications users} combined with the connection
  establishment procedure that characterizes WebRTC makes current peer-sampling
  solutions inefficient or simply unreliable.  In this paper, we address the
  limitations of current peer-sampling approaches by introducing \SPRAY, a novel
  peer-sampling protocol designed to avoid the constraints introduced by
  WebRTC. Unlike most recent peer-sampling approaches, \SPRAY has the ability to
  adapt its operation to networks that can grow or shrink very rapidly. Our
  experiments demonstrate the ability of \SPRAY to adapt to dynamic networks and
  highlight its efficiency improvements with respect to existing protocols. We
  built a real-time decentralized editor and ran experimentation involving
  uptill 600 communicating web browsers. The results highlight the benefits
  brought by \SPRAY's adaptiveness over communication complexity.
\end{abstract}

%%%%\category{C.2.1}{Computer Systems Organization}{Computer-Communication Networks}[Network Architecture and Design]

%%\terms{Network, algorithm, simulation}

\keywords{random peer sampling, network of browsers, WebRTC, large scale
  distributed applications.}

%%% Local Variables:
%%% mode: latex
%%% TeX-master: "../paper"
%%% End:
