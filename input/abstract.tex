
\begin{abstract}
  The introduction of WebRTC has opened a new playground for
  large-scale distributed applications running in browsers with
  browser-to-browser connections. if peer sampling protocols can be
  used to maintain devices connected, they do not adapt to networks
  that can grow or shrink very rapidely. Consequently, the resulting
  traffic can be oversized.  In this paper, we address the limitations
  of current peer-sampling approaches by introducing \SPRAY, a novel
  adaptive peer-sampling protocol that maintain the size of local
  routing table in the browser to the logarithm of the number of real
  participants. Our experiments demonstrate the ability of \SPRAY to
  adapt to dynamic networks, reducing the number of WebRTC connections
  needed to maintain browsers connected. In order to validate our
  proposal, we built a real-time decentralized editor running in
  browsers on laptops, tablets and mobile phones. We ran
  experimentation involving uptill 600 communicating web
  browsers. Experimentations demonstrate that SPRAY reduce
  significantly the network traffic according to the number of
  participants and save bandwitdh.
\end{abstract}


% \begin{abstract}
%   The introduction of WebRTC has opened a new playground for large-scale
%   distributed applications consisting of large numbers of directly-communicating
%   web browsers. In this context, gossip-based peer-sampling protocols appear as
%   a particularly promising tool thanks to their inherent ability to build
%   overlay networks that can cope with network dynamics. However, \TODO{the
%     dynamic nature of browser-to-browser communication} combined with the
%   connection establishment procedures that characterize WebRTC make current
%   peer-sampling solutions inefficient or simply unreliable.  In this paper, we
%   address the limitations of current peer-sampling approaches by introducing
%   \SPRAY, a novel peer-sampling protocol designed to avoid the constraints
%   introduced by WebRTC. Unlike most recent peer-sampling approaches, \SPRAY has
%   the ability to adapt its operation to networks that can grow or shrink very
%   rapidly. Our experiments demonstrate the ability of \SPRAY to adapt to dynamic
%   networks and highlight its efficiency improvements with respect to existing
%   protocols. We built a real-time decentralized editor and ran experimentation
%   involving uptill 600 communicating web browsers. Results highlight the
%   benefits brought by \SPRAY's adaptiveness over \TODO{protocols depending} on
%   it.
% \end{abstract}

%%%%\category{C.2.1}{Computer Systems Organization}{Computer-Communication Networks}[Network Architecture and Design]

%%\terms{Network, algorithm, simulation}

\keywords{random peer sampling, network of browsers, WebRTC, large scale
  distributed applications.}

%%% Local Variables:
%%% mode: latex
%%% TeX-master: "../paper"
%%% End:
