
\begin{abstract}

  Peer-sampling protocols are a fundamental mechanism for a number of
  large-scale distributed applications. The recent introduction of WebRTC
  facilitated the deployment of decentralized applications over a network of
  browsers. However, deploying existing peer-sampling protocols on top of WebRTC
  raises issues about their lack of adaptiveness to sudden bursts of popularity
  over a network that does not manage addressing nor routing. \SPRAY is a novel
  random peer-sampling protocol that dynamically, quickly, and efficiently
  self-adapts to the network size. Our experiments show the flexibility of
  \SPRAY and highlight its efficiency improvements at the cost of small
  overhead. We embedded \SPRAY in a real-time decentralized editor running in
  browsers and ran experiments involving up to 600 communicating web
  browsers. The results demonstrate that \SPRAY significantly reduces the
  network traffic according to the number of participants and saves bandwidth.

  % The introduction of WebRTC has opened a new playground for large-scale
  % distributed applications running in web browsers. Peer sampling protocols can
  % be used to maintain connectivity between devices. However, they do not adapt
  % to networks that can grow and shrink.  As a result, traffic can be oversized.
  % In this paper, we address the limitations of current peer sampling approaches
  % by introducing \SPRAY, a novel adaptive peer sampling protocol that maintains
  % local neighborhood tables that logarithmically scale with the total number of
  % network members. Our experiments demonstrate the ability of \SPRAY to adapt to
  % dynamic networks, reducing the number of WebRTC connections needed to maintain
  % browsers connected. To validate our proposal, we built a real-time
  % decentralized editor running in browsers, thus on laptops, tablets and mobile
  % phones. We ran experiments involving up to 600 communicating web browsers. The
  % results demonstrate that \SPRAY significantly reduces the network traffic
  % according to the number of participants and saves bandwidth.
\end{abstract}


% \begin{abstract}
%   The introduction of WebRTC has opened a new playground for large-scale
%   distributed applications consisting of large numbers of directly-communicating
%   web browsers. In this context, gossip-based peer-sampling protocols appear as
%   a particularly promising tool thanks to their inherent ability to build
%   overlay networks that can cope with network dynamics. However, \TODO{the
%     dynamic nature of browser-to-browser communication} combined with the
%   connection establishment procedures that characterize WebRTC make current
%   peer-sampling solutions inefficient or simply unreliable.  In this paper, we
%   address the limitations of current peer-sampling approaches by introducing
%   \SPRAY, a novel peer-sampling protocol designed to avoid the constraints
%   introduced by WebRTC. Unlike most recent peer-sampling approaches, \SPRAY has
%   the ability to adapt its operation to networks that can grow or shrink very
%   rapidly. Our experiments demonstrate the ability of \SPRAY to adapt to dynamic
%   networks and highlight its efficiency improvements with respect to existing
%   protocols. We built a real-time decentralized editor and ran experimentation
%   involving uptill 600 communicating web browsers. Results highlight the
%   benefits brought by \SPRAY's adaptiveness over \TODO{protocols depending} on
%   it.
% \end{abstract}

%%%%\category{C.2.1}{Computer Systems Organization}{Computer-Communication Networks}[Network Architecture and Design]

%%\terms{Network, algorithm, simulation}

%%% Local Variables:
%%% mode: latex
%%% TeX-master: "../paper"
%%% End:
