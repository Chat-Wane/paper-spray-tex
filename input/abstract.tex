
\begin{abstract}
  Recently, WebRTC allowed the connection establishment between web browsers,
  even with complex network settings such as firewall, proxies or Net Address
  Translation (NAT).  It eases the end-user access to distributed
  applications. However, several applications require to broadcast messages to
  a potentially large number of participants. Traditionally, gossiping
  algorithms handle such scalable broadcast by using random peer sampling
  protocols. The latter provides partial views (or random samples) of the whole
  network membership. However, the WebRTC protocol uses 3-way handshake
  connection establishment. Unfortunately, in this context, current approaches
  either lake of adaptiveness or robustness. In this paper, we introduce
  \SCAMPLON{}, a random peer sampling protocol
  \begin{inparaenum}[(i)]
  \item which provides optimal partial view size (i.e. logarithmically growing
    compared to the network size),
  \item which establishes connections solely using neighbor-to-neighbor
    interactions (i.e. constant time complexity of connections),
  \item which converges exponentially fast to a random graph.
  \end{inparaenum}
  As such, \SCAMPLON{} constitutes an improvement over state-of-the-art
  approaches in the traditional connection setup. This improvement becomes
  crucial in the 3-way handshake connections setup. It opens the gate to a wide
  variety of distributed applications directly usable within modern web
  browsers. The experimentation validates the aforementionned properties and
  highlights the behaviour of the graph generated by \SCAMPLON{}.
\end{abstract}


%%% Local Variables:
%%% mode: latex
%%% TeX-master: "../paper"
%%% End:
