
\section{Background and motivations}
\label{sec:background}

Random peer sampling protocols~\cite{jelasity2004peer} provide each peer with a
partial view $\mathcal{P}$ of the network membership $\mathcal{N}$. They
populate the partial views with the reference of peers chosen at random among
$\mathcal{N}$ following a uniform distribution using local knowledge
only. Their goal is to converge to an overlay network exposing similar
properties of random graphs~\cite{erdos1959random}. They efficiently provide
connectedness, robustness, messages dissemination etc. A wide variety of
gossip-based protocols use random peer sampling (e.g. topology
management~\cite{voulgaris2005epidemic, jelasity2009tman, dabek2004vivaldi}).

Representatives of random peer sampling protocols~\cite{voulgaris2005cyclon,
  eugster2003lightweight, tolgyeski2009adaptive} use a fixed-size partial view.
Thus, they have to know \emph{a priori} the maximum order of magnitude of the
network size to set their appropriate parameters. These decisions cannot be
safely retracted afterwards. For this reason, the partial views are commonly
oversized compared to the actual network size.

Figure (TODO) shows the theoretical number of arcs compared to
Cyclon~\cite{voulgaris2005cyclon}'s number of arcs. In particular, we can
observe that the Cyclon's partial view is indifferent to network changes.  When
the partial views of Cyclon are oversized, the members of the network suffer of
increased traffic to maintain the membership and redundancy in broadcast. When
the partial views of Cyclon are undersized, the network becomes weaker to
massive failures.

(TODO Motivating example including WebRTC and mobile devices)

\begin{figure*}
\centering
\begin{tikzpicture}
\matrix (m) [matrix of math nodes,row sep=4em,column sep=4em] {
\node(ss)[draw]{signaling}; & \node(p3)[draw]{p3}; \\
\node(p1)[draw]{p1}; & \node(p2)[draw]{p2}; \\
};
\path[->]
  (p2) edge[dashed] node[fill=white]{1:emit} (ss)
  (p3) edge[dashed] node[fill=white,bend left]{2:pull} (ss)
  (p3) edge[dashed, bend right] node[fill=white]{3:accept} (ss)
  (p2) edge[dashed,bend left] node[fill=white]{4:pull} (ss)
  (p3) edge[<->,thick] node[fill=white,right]{5:connected} (p2);
\end{tikzpicture}%
\begin{tikzpicture}
\matrix (m) [matrix of math nodes,row sep=4em,column sep=4em] {
\node(ss)[draw]{signaling}; & \node(p3)[draw]{p3}; \\
\node(p1)[draw]{p1}; & \node(p2)[draw]{p2}; \\
};
\path[->]
  (p1) edge[dashed,bend left] node[fill=white]{1:emit} (p2)
  (p2) edge[dashed,bend left] node[fill=white,left]{2:emit/p1} (p3)
  (p3) edge[dashed,bend left] node[fill=white,right]{3:accept/p1} (p2)
  (p2) edge[dashed,bend left] node[fill=white]{4:accept} (p1)
  (p1) edge[<->,thick] (p2)
%  (p1) edge[<->,thick,bend left] (p3)
  (p2) edge[<->,thick]  (p3);

\end{tikzpicture}%
\begin{tikzpicture}
\matrix (m) [matrix of math nodes,row sep=4em,column sep=4em] {
\node(ss)[draw]{signaling}; & \node(p3)[draw]{p3}; \\
\node(p1)[draw]{p1}; & \node(p2)[draw]{p2}; \\
};
\path[->]
  (p1) edge[<->,thick] (p2)
  (p1) edge[<->,thick] (p3)
  (p2) edge[<->,thick]  (p3);
\end{tikzpicture}
\caption{Connection establishement using WebRTC.}
\end{figure*}

WebRTC is able to establish a data channel between two browsers even
with complex network settings such as firewall, proxies or Net Address
Translation (NAT). Nevertheless, the connection establishment protocol
differs from traditional setting: it uses a three-way handshake
connection set up.  Thus, a subscription message must be
\begin{inparaenum}[(1)]
\item emitted at Peer $p$,
\item accepted at Peer $q$,
\item completed at Peer $p$.
\end{inparaenum}
  
Therefore, when a peer needs to establish a connection to another
peer, a message must travel back and forth. One could use a dedicated
server to establish the necessary dialog~\cite{peerjs}. However, this
solution does not scale considering the fact that each peer in the
network must dynamically create and remove connections over time. On
the other hand, using the built network itself distributes the load of
these connections establishment among the members.


\begin{problem}
  Let $t$ be an arbitrary time frame, let $\mathcal{N}^t$ be the network
  membership at that given time $t$ and let $\mathcal{P}_x^t$ be the partial
  view of peer $p_x \in \mathcal{N}^t$.  A cost-efficient random peer sampling,
  especially when three-way handshaking is involved, should provide the
  following best-case properties:
  \begin{center}
    Partial view size: \hfill
    $\forall p_x \in \mathcal{N}^t,\, |\mathcal{P}_x^t| = \Theta (\ln
    |\mathcal{N}^t|)$
  \end{center}
  \begin{center}
    Connection establishment: \hfill $O(1)$
  \end{center}
  \begin{center}
    Convergence speed: \hfill $\Theta(\exp \, t^{-1})$
  \end{center}
\end{problem}

%%% Local Variables:
%%% mode: latex
%%% TeX-master: "../paper"
%%% End:
