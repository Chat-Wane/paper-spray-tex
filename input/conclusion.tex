
\section{Conclusion and perspectives}
\label{sec:conclusion}

In this paper, we introduced a random peer sampling approach called
\SCAMPLON{}.  This new approach, inspired of two state-of-the-art approaches,
namely \SCAMP{} and \CYCLON{}, provide the best of both worlds:
\begin{inparaenum}[(i)]
\item logarithmically growing partial views reflecting the global network size,
\item constant time complexity on connection establishments using solely
  neighbor-to-neighbor interactions,
\item exponentially fast convergence a random graph.
\end{inparaenum}
As such, \SCAMPLON{} constitutes an improvement over state-of-the-art
approaches in the traditional context of 1-way connection establishments. In
addition, the improvement becomes crucial in the 3-way handshake connection
set-up which is increasingly important since recent technologies (e.g. WebRTC)
allow peer-to-peer connections between modern web browsers. Random peer
sampling protocols constitute the ground of many decentralized applications.
Thus, we expect that decentralized applications (requiring scalable broadcast
among other) will flourish in web browsers, while they were previously
quartered to standalone applications.

Future work includes an analysis of doubles in the \SCAMPLON{}'s partial views.
Intuitively, these artifacts do not exist in infinite random graphs. However,
in small networks, they are much more likely to exist. We intend to measure
these redundant identities and to what extent they become a threat to the
random topology.  We also want to extend the adaptive partial views to topology
managers that rely on random peer sampling protocols such as
T-Man~\cite{jelasity2009tman} or Vicinity~\cite{voulgaris2005epidemic}. For
instance, it is possible to build a topology based on preferences. Current
approaches have a fixed-size view of the most closely related neighbors. Once
again, it may lead to waste in efficiency. On the other hand, if the view size
could adapt to the size of a cluster (if the topology creates disjoint
clusters), it would improve the traffic, robustness, etc. Finally, future work
includes a Javascript implementation using WebRTC. While experimentation of
this paper shows the results of Peersim simulations, the real implementation
would open the gate to emulations, and even real peer-to-peer distributed and
decentralized applications.

%%% Local Variables:
%%% mode: latex
%%% TeX-master: "../paper"
%%% End:
