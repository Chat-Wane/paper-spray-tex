
\section{Conclusion and perspectives}
\label{sec:conclusion}

WebRTC opened a new playground for large-scale distributed
applications deployed on a network of browsers. Browsers as
infrastructure makes large-scale distributed applications easy to
deploy for end-users. As a core component of many large-scale
distributed applications, we pointed out limitations of current
peer-sampling protocols in term of adaptivity or reliability for a
deployment on top of WebRTC. 

In this paper, we described \SPRAY, an adaptative random peer sampling
approach designed to fit the WebRTC constraints.  \SPRAY provides:
\begin{inparaenum}[(i)]
\item logarithmically growing partial views reflecting the global network size,
\item constant time complexity on connection establishments using solely
  neighbor-to-neighbor interactions,
% \item an exponentially fast convergence to an overlay network exposing
%   properties similar to random graphs.
\end{inparaenum}

% Synthesis of experiments
In experiments, we demonstrated how \SPRAY adaptivity improves random
peer sampling performances when network size is changing. In
particular, the average shortest path length scales better, the
in-degree evolves with the network size, and it converges faster.  We
also demonstrated that \SPRAY stays robust to massive failures. \SPRAY
and \CYCLON are quite similar when the network size is optimal for
\CYCLON. However, \SPRAY saves connections when \CYCLON is oversized
and is more robust when \CYCLON is undersized. This adaptivity comes
at the price of duplicates in partial views. Although, the simulations
supported by theoretical analysis shows that number of duplicates remains very low and becomes negligible in large networks.


% As such, \SPRAY constitutes an improvement over state-of-the-art
% approaches in the traditional context of 1-way connection establishments. In
% addition, the improvement becomes crucial in the 3-way handshake connection
% set-up which is increasingly important since recent technologies (e.g. WebRTC)
% allow peer-to-peer connections between modern web browsers. Random peer
% sampling protocols constitute the ground of many decentralized applications.
% Thus, we expect that decentralized applications (requiring scalable broadcast
% among other) will flourish in the web browsers, while they were previously
% quartered to standalone applications.


Future work includes a Javascript implementation of \SPRAY. An in
browser implementation opens the gate to emulations, and even
real peer-to-peer distributed and decentralized applications.

Future work also includes investigations on topology managers such as
T-Man~\cite{jelasity2009tman} or
Vicinity~\cite{voulgaris2005epidemic}. Indeed, they traditionally rely
on random peer sampling approaches using fixed-size partial
view. Thus, they maintain a fixed-size view of their most closely
related neighbors using a ranking function. With \SPRAY, it is
possible to extend their behavior to use dynamic partial views. If
the view size could adapt to the size of a cluster (if the topology
creates disjoint clusters), it would improve the traffic, robustness,
etc.

%%% Local Variables:
%%% mode: latex
%%% TeX-master: "../paper"
%%% End:
