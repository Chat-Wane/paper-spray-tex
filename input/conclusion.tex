
\section{Conclusion and perspectives}
\label{sec:conclusion}

%%\TODO{A lot to change. Includes perspective with security, maybe problems of
%%  merging etc.}

% WebRTC opened a new playground for large-scale distributed
% applications deployed on a network of browsers. Browsers as an
% infrastructure ease the deployment of large-scale distributed
% applications for end-users. Yet, WebRTC exacerbates the limitations,
% in term of adaptiveness and reliability, of a core component of many
% large-scale distributed applications: peer-sampling protocols.

In this paper, we described \SPRAY, an adaptive-by-design random peer-sampling
approach designed to fit WebRTC's constraints.  \SPRAY provides:
\begin{inparaenum}[(i)]
\item logarithmically growing partial views reflecting the global network size,
\item constant time complexity on connection establishments using solely
  neighbor-to-neighbor interactions.
% \item an exponentially fast convergence to an overlay network exposing
%   properties similar to random graphs.
\end{inparaenum}
Our experiments demonstrate how \SPRAY's adaptiveness improves the performance
of random peer-sampling when the size of the network changes. In particular, the
average shortest path length scales better and the in-degree evolves with the
network size.
%, and it converges faster.
Adaptiveness comes at the price of duplicates in the partial views. However, our
simulations supported by theoretical analysis show that the number of duplicates
remains very low and becomes negligible in large networks.

To highlight the improvements brought by \SPRAY, we demonstrated how a
broadcast protocol can take advantage of its adaptiveness to adjust
the fanout and handle a sudden burst of popularity. We also built
\CRATE, a decentralized collaborative editor directly accessible in
web browsers with a full implementation of \SPRAY on top of WebRTC. We
launched experiments involving up to 600 browsers showing the %that message
%dissemination highly
benefits of adaptive neighborhoods.

\SPRAY makes building scalable decentralized applications in browsers
easy and accessible.  Developers do not require to foresee the network
size targeted by their applications. Just the same as for broadcast,
many topology optimization protocols (e.g. about geolocalization,
latency, or preferences) rely on random peer-sampling protocol such as
the generic algorithms T-Man~\cite{jelasity2009tman} and
Vicinity~\cite{voulgaris2005epidemic}. \SPRAY allows such protocols to
benefit from its self-adjusting partial views.

Peer-sampling protocols build overlay networks. These overlay networks do not
necessarily reflect the underlying topology made of routers. Using WebRTC, a
connection establishment may fail depending on network
configurations. Section~\ref{subsec:leaving} states that such failures should be
handled by replacing failed links by duplicates to known peers. As future work,
it would be interesting to see if this strategy leads to an overlay network that
self-adapts to the real underlying topology.

% Future work includes investigations on topology managers such as
% T-Man~\cite{jelasity2009tman} or
% Vicinity~\cite{voulgaris2005epidemic}. Indeed, they traditionally rely
% on random peer sampling approaches using fixed-size partial
% view. Thus, they maintain a fixed-size view of their most closely
% related neighbors using a ranking function. With \SPRAY, we could
% extend their behavior to use dynamic partial views.

%%% Local Variables:
%%% mode: latex
%%% TeX-master: "../paper"
%%% End:
