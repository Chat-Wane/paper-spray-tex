
\section{Conclusion and perspectives}
\label{sec:conclusion}

The introduction of WebRTC has opened a new playground for large-scale
distributed applications consisting of large numbers of directly-communicating
web browsers. Nevertheless, this requires to deploy adaptative peer-sampling
protocols compatible with the WebRTC connection constraints.

In this paper, we described \SPRAY, an adaptative random peer sampling approach
designed to fit the WebRTC constraints.  \SPRAY provides:
\begin{inparaenum}[(i)]
\item logarithmically growing partial views reflecting the global network size,
\item constant time complexity on connection establishments using solely
  neighbor-to-neighbor interactions,
\item an exponentially fast convergence to an overlay network exposing
  properties similar to random graphs.
\end{inparaenum}

% Synthesis of experiments
In experiments, we demonstrated that \SPRAY gently adapts its partial views to
the network size at the price of duplicates. Although, the simulations
supported by theoretical analysis shows that these duplicates stay few in
number and becomes negligible in large networks. Contrarily to \CYCLON, a
representative of fixed-size partial view approaches, \SPRAY automatically
scales to the network size in terms of robustness and efficiency. In
particular, the average shortest path length scales better, the in-degree
evolves with the network size, and it converges faster.  We also demonstrated
that \SPRAY stays robust to massive failures.

% As such, \SPRAY constitutes an improvement over state-of-the-art
% approaches in the traditional context of 1-way connection establishments. In
% addition, the improvement becomes crucial in the 3-way handshake connection
% set-up which is increasingly important since recent technologies (e.g. WebRTC)
% allow peer-to-peer connections between modern web browsers. Random peer
% sampling protocols constitute the ground of many decentralized applications.
% Thus, we expect that decentralized applications (requiring scalable broadcast
% among other) will flourish in the web browsers, while they were previously
% quartered to standalone applications.


Future work includes a Javascript implementation of \SPRAY. An in
browser implementation opens the gate to emulations, and even
real peer-to-peer distributed and decentralized applications.

Future work also includes investigations on topology managers such as
T-Man~\cite{jelasity2009tman} or
Vicinity~\cite{voulgaris2005epidemic}. Indeed, they traditionally rely
on random peer sampling approaches using fixed-size partial
view. Thus, they maintain a fixed-size view of their most closely
related neighbors using a ranking function. With \SPRAY, it is
possible to extend their behavior to use dynamic partial views. If
the view size could adapt to the size of a cluster (if the topology
creates disjoint clusters), it would improve the traffic, robustness,
etc.

%%% Local Variables:
%%% mode: latex
%%% TeX-master: "../paper"
%%% End:
