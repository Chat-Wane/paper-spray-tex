
\section{Introduction}

Recently, WebRTC~\cite{webrtc} allowed the connection establishment between
web browsers, even with complex network settings such as firewall, proxies or
Net Address Translation (NAT).  Some attractive applications such as Firefox
Hello~\cite{firefoxhello} or WebTorrent~\cite{webtorrent} demonstrate how
video-conferencing or file-sharing directly within browsers becomes simple of
access to end-users without third-party providers. However, several distributed
applications such as distributed collaborative editing (REF) require the
ability to broadcast messages to a potentially large number of
participants. Traditionally, gossiping algorithms (REF) achieve such
broadcast. Unfortunately, the current gossiping algorithms in a WebRTC network
raises reliability and efficiency issues.

Gossiping requires that each participant maintains connections with a uniform
sample of other participants~\cite{jelasity2004peer}. The sample, called
partial view, can grow logarithmically compared to the number of participants
and yet providing connectivity with high probability. Still, the neighbouring
must be constantly renewed in order to prevent network partitions. In WebRTC,
establishing a browser-to-browser connection requires a three-way handshake
which is much more costly and likely to fail compared to traditional
setting. Consequently, existing gossiping algorithms will either be resource
intensive, or fail to maintain a connected network.

Knowingly, the main challenge is about
\begin{inparaenum}[(i)]
\item quickly converging to a random graph.
\item Keeping an optimal number of connections alive. To provide connectedness
  in random graph, each peer must have $\Theta(\text{ln} |\mathcal{N}|)$
  neighbours (REF) where $\mathcal{N}$ is the set of network members.
\item Keeping an optimal complexity on connection establishment.
\end{inparaenum}

% The \CYCLON{} protocol~\cite{voulgaris2005cyclon} provides each peer with a
% fixed size partial view.  Each element in this view has an age which allows
% quick withdrawing of left peers. Nevertheless, the user must foresee the
% maximum size of the network to set the appropriate view size. As such, the
% views are generally oversized generating too much resource consumption in a
% WebRTC network. On the other hand, the \SCAMP{}
% protocol~\cite{ganesh2001scamp,ganesh2003peer} incrementally builds the
% partial views of peers at join. Thus, each peer has a partial view size
% logarithmically growing compared to the global size of the
% network. Nevertheless, a peer may connect to peers at several hops away from
% him. Hence, the connections are more likely to fail.

In this paper, we introduce \SCAMPLON{}, a random peer sampling protocol
\begin{inparaenum}[(i)]
\item which provides each peer with a partial view logarithmically growing
  compared to the global network size,
\item which quickly converges to a random graph,
\item which can handle three-way handshake.
\end{inparaenum}
Experiments shows that it outperforms the state-of-the-art approaches both in
one-way connections and three-way handshake connections.

The rest of this paper is organised as follow: Section~\ref{sec:background}
introduces the necessary background to understand \SCAMPLON{} and highlights
our motivations. Section~\ref{sec:proposal} details the \SCAMPLON{} protocol.
Section~\ref{sec:experiments} shows the properties of \SCAMPLON{} and compare
them to state-of-the-art random peer sampling
approaches. Section~\ref{sec:relatedwork} reviews the related work. We conclude
and discuss about perspective in Section~\ref{sec:conclusion}.

%%% Local Variables:
%%% mode: latex
%%% TeX-master: "../paper"
%%% End:
