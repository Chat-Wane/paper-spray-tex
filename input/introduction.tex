
\section{Introduction}

Peer sampling constitutes a fundamental mechanism for many large-scale
distributed applications both on the
cloud~\cite{decandia2007dynamo} and in a peer-to-peer
setting. Services such as dissemination~\cite{eugster2003lightweight,
  tolgyeski2009adaptive}, aggregation~\cite{jelasity2004epidemic} and
network management~\cite{jelasity2009tman, voulgaris2005epidemic} have
been based on peer sampling and the recent introduction of
WebRTC\footnote{\url{http://www.webrtc.org/}} opens the opportunity to
deploy such applications on web browsers.
% that can run on laptops, desktops and mobile devices.
In this context, WebRTC drastically simplifies deployment even within complex
network systems that utilize firewalls, proxies and Net Address Translation
(NAT).

Unfortunately, WebRTC has several constraints that make existing peer sampling
services inefficient or unreliable. Browsers can run on small devices in mobile
networks. Hence, keeping the number of connections as low as possible is a major
requirement. It becomes crucial when these connections are actively used to
broadcast messages. Yet, peer sampling services such as
\CYCLON~\cite{voulgaris2005cyclon} do not adapt their operation to the real
network size. \TODO{Even less, considering the dynamicity of web applications
  users}. \SCAMP~\cite{ganesh2003peer} is an adaptive peer sampling service but
it becomes costly and unreliable in WebRTC context. To build applications
running on networks of browsers, we need a reliable peer sampling service
scaling to small groups, large groups, and groups of fluctuating size.

In this paper, we introduce \SPRAY, a random peer sampling protocol inspired by
both \SCAMP~\cite{ganesh2003peer}
and \CYCLON~\cite{voulgaris2005cyclon}. \TODO{Compared to the state-of-the-art},
\begin{inparaenum}[(i)]
\item \SPRAY dynamically adapts the neighborhood of each peer. Thus, the number
  of connections is logarithmically scaling to the network size.
\item \SPRAY only uses neighbor-to-neighbor interactions to establish
  connections. Thus, the connections are established in constant time.
\item \SPRAY quickly converges to a topology exposing properties similar to
  those of random graphs. Thus, the network becomes robust to massive failures,
  quickly disseminates information etc.
\item Experiments show the adaptiveness of \SPRAY and highlight its efficiency
  improvement at the cost of little overhead.
\end{inparaenum}

To highlight the improvement brought by \SPRAY we built \CRATE, a real-time
decentralized editor directly running in web
browsers\footnote{\url{http://chat-wane.github.io/CRATE/}}. This real-life use
case shows that \TODO{information dissemination} highly benefits from the
adaptiveness of \SPRAY. The experiments running on Grid'5000 and involving
uptill 600 editors confirm \TODO{scalability and efficiency}.

The rest of this paper is organized as follows: Section~\ref{sec:relatedwork}
reviews related work. Section~\ref{sec:proposal} details the \SPRAY protocol and
compares it to state-of-the-art. Section~\ref{sec:proposal2} provides a
real-life use case about real-time decentralized editing in browsers. We
conclude and discuss about the perspective in Section~\ref{sec:conclusion}.

%%% Local Variables:
%%% mode: latex
%%% TeX-master: "../paper"
%%% End:
