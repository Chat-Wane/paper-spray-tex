
\section{Introduction}

Peer-sampling
protocols~\cite{voulgaris2005cyclon,jelasity2007gossip,tolgyeski2009adaptive}
constitute a fundamental mechanism for a number of large-scale
distributed applications both on the Cloud~\cite{decandia2007dynamo}
and in a peer-to-peer
setting~\cite{Frey09Middleware,voulgaris2005sub,wuhib2009robust}. By
providing each node with a continuously changing partial view of the
network, they make applications resilient to churn~\cite{bertier-d2ht}
and inherently load balancing~\cite{Frey09DSN}. In the context of
video streaming, for example, a peer-sampling protocol makes it
possible to distribute the streaming load over all peers without
requiring the creation and the maintenance of rigid structures like
multiple trees~\cite{Frey09DSN, monod:THESIS}.  

% made browser-to-browser communications easy even
% within complex network environments that involve firewalls, proxies,
% Net Address Translation (NAT) and mobile networks. A simple click on a
% http link in the browser launches a browser-based video-conferencing
% application as in Firefox Hello, a torrent download browser-based application as in
% webtorrent~\cite{webtorrent}, or a decentralized collaborative
% real-time editor as in CRATE~\cite{nedelec2016crate}. Browsers can be
% seen as an application containter and WebRTC as a communication layer
% for distributed applications deployed in browsers. Users start an
% application in their browsers, create a session URL to allow remote
% access and notify other users. Remote participants have just to click
% on this link to deploy the application in their browsers and join the session.

% Many large scale distributed applications relies on gossip-based peer
% sampling protocols such as \CYCLON~\cite{voulgaris2005cyclon} to
% enable information dissemination~\cite{eugster2003lightweight,
%   tolgyeski2009adaptive}, aggregation~\cite{jelasity2004epidemic} or
% network management~\cite{jelasity2009tman, voulgaris2005epidemic}.


The recent introduction of WebRTC~\cite{webrtc} has renewed the
research interest in this kind of decentralized applications with a
number of proposals for applications like decentralized video
streaming~\cite{hivejs,smoothcache2}, content-delivery
networks~\cite{Zhang:2013:MBC:2465351.2465379} or real-time
collaborative editors~\cite{nedelec2016crate}. However, deploying
existing peer sampling protocols on top of WebRTC raises important
technical challenges. 
\begin{inparaenum}[(1)]
\item WebRTC does not manage addressing nor routing; this makes
  connection establishement much more costly than on IP network and
  more likely to fail. 
\item Browsers run on desktop, laptops and mobile phones. This
  requires protocols that reduce resource consumption as much as
  possible.
\item The ability to launch WebRTC sessions through simple HTTP makes
  applications subject to sudden bursts of popularity.  % Suppose
  % massive online lecture platforms allow students to share their
  % notes. Many lectures run in parallel involving various number of
  % students, i.e., from few to thousands. Also, even during the
  % editing
  % session the audience is subject to significant changes in size,
  % going from thousands to hundreds.
\end{inparaenum}
Consider the example of a user who is streaming a video directly from
his mobile phone to some of his friends. The user suddenly witnesses
some dramatic event, and his friends spread the news by twitting the
stream's address. Instantly a huge number of users connect and start
watching the stream on their laptops and phones. The streaming
mechanisms, and the protocols it relies on must be able to adapt to
this sudden burst of popularity, maintaining their quality of service,
while being able to return to their initial configuration when the
popularity burst subsides. 

% Publishing the
%   link to a WebRTC video stream on Twitter can create the buzz and
%   generate massive joins for an unpredictable period of time.

Unfortunately, existing peer-sampling protocols
% Pascal: IMHO not useful here
%, which lie at the bases of several video streaming
%solutions~\cite{Frey09Middleware,Abeni2009,smoothcache20} 
lack this capability. On the one hand, the most popular approaches like
\CYCLON~\cite{voulgaris2005cyclon}, and the RPS protocol
family~\cite{jelasity2007gossip} provide application nodes with fixed sized
views of the network based on parameters that have to be configured at
deployment time. This forces developers to oversize partial views to handle
potential bursts, either wasting resources, or not provisioning for large-enough
settings. On the other, \SCAMP~\cite{ganesh2003peer} features some form of
adaptation but falls short in the context of WebRTC applications. \SCAMP
maintains partial views of size $\log(n)+k$, $n$ being the number of nodes, and
$k$ being a constant. But its connection establishment process does not support
the connection failures that often occur in WebRTC.
%  Unfortunatly, deploying Scamp on
% top of WebRTC fails due to the connection estabilishement failures of
% WebRTC. Another approach is just to oversize partial views and use
% protocol as Cyclon~\cite{voulgaris2005cyclon}. Unfortunatly, this
% approach consumes more ressources and create and renew periodically
% more connections than required. With this approach, small networks
% will end up paying the price of the largest one. It does not also
% provide a way to compute a fan-out that will ensure high delivery rate
% in any situation.
Finally, it would be possible to periodically estimate the real number
of network nodes using an aggregation
protocol~\cite{montresor2004robust}, and to reconfigure \CYCLON, or RPS
solutions based on this. However, this aggregation protocol would have
to run quite frequently, resulting in significant network overhead
to anticipate a popularity burst that may never happen. 

% The scientific challenge consists in finding a way to adapt to the
% network size without actually measuring its number of members as
% proposed by Scamp, but running on top of WebRTC.

In this paper, we address the challenge of a dynamically adaptive
peer-sampling, by introducing \SPRAY, a novel random peer-sampling
protocol inspired by both \SCAMP~\cite{ganesh2003peer} and
\CYCLON~\cite{voulgaris2005cyclon}. \SPRAY improves the
state-of-the-art in several ways.
\begin{inparaenum}[(i)]
\item It dynamically adapts the neighborhood of each peer. Thus, the
  number of connections scales logarithmically with the network size.
\item It only uses neighbor-to-neighbor interactions to establish
  connections. Thus, the process takes constant time.
\item It quickly converges to a topology with properties similar to
  those of random graphs. Thus, the network becomes robust to massive
  failures, quickly disseminates information etc.
\item Experiments show the flexibility of \SPRAY and highlight its
  efficiency improvement at the cost of little overhead.
\end{inparaenum}

\SPRAY not only relieves developers from having to foresee the number
and range of users of their distributed applications, but most
importantly it allows applications to adapt to sudden burst in
popularity. We demonstrate the effectiveness of \SPRAY by simulation
in the context of a large-scale flash-crowd scenario, as well as on a
real use case using \CRATE~\cite{nedelec2016crate}, deployed on
Grid'5000. Our results demonstrate that \SPRAY significantly reduces
network traffic with respect to a standard peer sampling protocol,
while adapting to the current size of the network. 

% , a real-time decentralized editor directly
% running in web browsers, thus on laptops, tablets, or mobile
% phones. We experiment with \CRATE by deploying 600 editors on

The rest of this paper is organized as follows: Section~\ref{sec:relatedwork}
reviews the related work. Section~\ref{sec:problem} states the scientific
problem. Section~\ref{sec:proposal} details the \SPRAY
protocol. Section~\ref{sec:experimentation} presents experimentation results of
\SPRAY and compares them to state-of-the-art. Section~\ref{sec:use-case} details
our experiment with \CRATE a real-time collaborative editor running in
browsers. We conclude in Section~\ref{sec:conclusion}.

%%% Local Variables:
%%% mode: latex
%%% TeX-master: "../paper"
%%% End:
