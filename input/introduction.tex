
\section{Introduction}

% Peer sampling constitutes a fundamental mechanism for many large-scale
% distributed applications both on the
% cloud~\cite{decandia2007dynamo} and in a peer-to-peer
% setting. Services such as dissemination~\cite{eugster2003lightweight,
%   tolgyeski2009adaptive}, aggregation~\cite{jelasity2004epidemic} and
% network management~\cite{jelasity2009tman, voulgaris2005epidemic} have
% been based on peer sampling and the recent introduction of
% WebRTC\footnote{\url{http://www.webrtc.org/}} opens the opportunity to
% deploy such applications on web browsers.
% that can run on laptops, desktops and mobile devices.

WebRTC~\cite{webrtc} makes browser-to-browser communication easy even within
complex network environments that involve firewalls, proxies, Net Address
Translation (NAT) and mobile networks. As a result, it opens the door to a new
generation of large-scale distributed applications that exploit only
browser-to-browser connections. Among the basic building blocks for such
distributed applications, we find gossip-based peer sampling protocols such as
\CYCLON~\cite{voulgaris2005cyclon}, which can build and maintain reliable
overlay networks in dynamic environments.  However, \CYCLON does not effectively
adapt to networks that grow or shrink and forces peers to maintain unnecessary
connections.  For instance, it might require a mobile phone to maintain 10
WebRTC connections with other remote browsers when only 6 would be enough. This
yields unnecessary network traffic, particularly when applications employ these
connections to broadcast messages.

In this paper, we introduce \SPRAY, a random peer sampling protocol
inspired by both \SCAMP~\cite{ganesh2003peer} and
\CYCLON~\cite{voulgaris2005cyclon}. \SPRAY improves the state-of-the-
art in several ways. 
\begin{inparaenum}[(i)]
\item It dynamically adapts the neighborhood of each peer. Thus, the
  number of connections scales logarithmically with the network size.
\item It only uses neighbor-to-neighbor interactions to establish
  connections. Thus, the process takes constant time.
\item It quickly converges to a topology with properties similar to
  those of random graphs. Thus, the network becomes robust to massive
  failures, quickly disseminates information etc.
\item Experiments show the flexibility of \SPRAY and highlight its
  efficiency improvement at the cost of little overhead.
\end{inparaenum}

To demonstrate the effectiveness of \SPRAY, we also introduce \CRATE, a
real-time decentralized editor directly running in web browsers, thus on
laptops, tablets, or mobile phones. We experiment with \CRATE by deploying 600
editors on Grid'5000. Our results demonstrate that \SPRAY significantly reduces
network traffic with respect to a standard peer sampling protocol, and adapts to
the number of participants.

The rest of this paper is organized as follows:
Section~\ref{sec:relatedwork} reviews the related
work. Section~\ref{sec:proposal} details the \SPRAY
protocol. Section~\ref{subsec:experiments} presents experimentation
results of \SPRAY and compares them to
state-of-the-art. Section~\ref{sec:use-case} details our experiment
with \CRATE a real-time collaborative editor running in browsers. We
conclude and discuss about the perspective in
Section~\ref{sec:conclusion}.

%%% Local Variables:
%%% mode: latex
%%% TeX-master: "../paper"
%%% End:
