
\section{Introduction}

Recently, WebRTC~\cite{webrtc} allowed the connection establishment between
web browsers, even with complex network settings such as firewall, proxies or
Net Address Translation (NAT).  Some attractive applications such as Firefox
Hello~\cite{firefoxhello} or WebTorrent~\cite{webtorrent} demonstrate how
video-conferencing or file-sharing directly within browsers becomes simple of
access to end-users without third-party providers. However, several distributed
applications such as distributed collaborative editing (REF) require the
ability to broadcast messages to a potentially large number of
participants. Traditionally, gossiping algorithms (REF) achieve such
broadcast. Unfortunately, the current gossiping algorithms in a WebRTC network
raises reliability and efficiency issues.

Gossiping requires that each participant maintains connections with a uniform
sample of other participants~\cite{jelasity2004peer}. The sample, called
partial view, can grow logarithmically compared to the number of participants
and yet providing connectivity with high probability. Still, the neighborhood
must be constantly renewed in order to prevent network partitions. In WebRTC,
establishing a browser-to-browser connection requires a three-way handshake
which is much more costly and likely to fail compared to traditional
setting. Consequently, existing gossiping algorithms (REFs) will either be
resource intensive, or fail to maintain a connected network.

Knowingly, the main challenge is about
\begin{inparaenum}[(i)]
\item keeping an optimal $\Theta(\text{ln}\, |\mathcal{N}|)$ on the number of
  connections alive (REF) (where $\mathcal{N}$ is the set of network members),
\item keeping an optimal $O(1)$ complexity on connection establishments,
\item keeping an optimal $O(\exp \, t^{-1})$ convergence speed to a random
  graph (where $t$ is an arbitrary time frame).
\end{inparaenum}

In this paper, we introduce \SCAMPLON{}, a random peer sampling protocol which
provides the desired properties described in the aforementioned problem
statement.
\begin{inparaenum}[(i)]
\item It dynamically adapts the neighborhood of each peer. Thus, the number of
  connections grows logarithmically compared to the size of the network.
\item It only uses neighbor-to-neighbor interactions to establish the
  connections. Thus, the connections are established in constant time.
\item It converges exponentially fast to a random graph. Thus, the network
  becomes quickly resilient to failure, can broadcast messages efficiently etc.
\end{inparaenum}
Knowingly, it outperforms state-of-the-art approaches in the traditional
connection set-up. This improvement is crucial in the three-way connection
set-up.


The rest of this paper is organized as follow: Section~\ref{sec:background}
introduces the necessary background to understand \SCAMPLON{} and highlights
our motivations. Section~\ref{sec:proposal} details the \SCAMPLON{} protocol.
Section~\ref{sec:experiments} shows the properties of \SCAMPLON{} and compare
them to state-of-the-art random peer sampling
approaches. Section~\ref{sec:relatedwork} reviews the related work. We conclude
and discuss about perspective in Section~\ref{sec:conclusion}.

%%% Local Variables:
%%% mode: latex
%%% TeX-master: "../paper"
%%% End:
