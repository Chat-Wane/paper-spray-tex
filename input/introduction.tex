
\section{Introduction}

% Peer sampling constitutes a fundamental mechanism for many large-scale
% distributed applications both on the
% cloud~\cite{decandia2007dynamo} and in a peer-to-peer
% setting. Services such as dissemination~\cite{eugster2003lightweight,
%   tolgyeski2009adaptive}, aggregation~\cite{jelasity2004epidemic} and
% network management~\cite{jelasity2009tman, voulgaris2005epidemic} have
% been based on peer sampling and the recent introduction of
% WebRTC\footnote{\url{http://www.webrtc.org/}} opens the opportunity to
% deploy such applications on web browsers.
% that can run on laptops, desktops and mobile devices.

WebRTC\footnote{\url{http://www.webrtc.org/}} drastically simplified
the setup browser-to-browser connections even within complex network
environment involving firewalls, proxies, Net Address Translation
(NAT) and mobile networks. By the way, it opens a new playground to
deploy large-scale distributed applications running in browsers with
just browser-to-browser connection. if peer sampling protocols such as
\CYCLON~\cite{voulgaris2005cyclon} can be used to maintain browsers
connected, it does not adapt to networks that can grow or shrink
rapidely. For instance, a user on his mobile phone must maintain 10
WebRTC connections with other remote browsers when only 6 are
enough. This makes a big difference when these connections are
actively used to broadcast messages generating oversized traffic on
the network.

In this paper, we introduce \SPRAY, a random peer sampling protocol inspired by
both \SCAMP~\cite{ganesh2003peer}
and \CYCLON~\cite{voulgaris2005cyclon}. Compared to the state of art,
\begin{inparaenum}[(i)]
\item \SPRAY dynamically adapts the neighborhood of each peer. Thus, the number
  of connections grows logarithmically compared to the size of the network.
\item \SPRAY only uses neighbor-to-neighbor interactions to establish
  connections. Thus, the connections are established in constant time.
\item \SPRAY quickly converges to a topology exposing properties similar to
  those of a random graph. Thus, the network becomes robust to massive
  failures, quickly disseminates information etc.
\item Experiments show the adaptiveness of \SPRAY and highlight its efficiency
  improvement at the cost of little overhead, compared to a representative of
  not adaptive approaches, namely \CYCLON.
\end{inparaenum}

To highlight the improvement brought by \SPRAY we built \CRATE, a
real-time decentralized editor directly running in web
browsers\footnote{\url{http://chat-wane.github.io/CRATE/}} on laptops,
tablets and mobile phones. The experiments running on Grid'5000 and
involving uptill 600 editors demonstrated how SPRAY significantly
reduce the network traffic according to the number of
participants. 

The rest of this paper is organized as follows:
Section~\ref{sec:relatedwork} reviews the related
work. Section~\ref{sec:proposal} details the \SPRAY
protocol. Section~\ref{subsec:experiments} presents experimentation
results of of \SPRAY and compares them to
state-of-the-art. Section~\ref{sec:use-case} details our experiment
with \CRATE a real-time collaborative editor running in browsers. We
conclude and discuss about the perspective in
Section~\ref{sec:conclusion}.

%%% Local Variables:
%%% mode: latex
%%% TeX-master: "../paper"
%%% End:
