
\section{Introduction}

Peer-sampling service is a fundamental mechanism for many large scale
distributed applications including information dissemination,
aggregation or network management~\cite{jelasity2007gossip}. Recently,
WebRTC~\cite{webrtc} opened the opportunity to deploy large scale
distributed application on browsers that can run on laptops, desktops,
tablets or mobile phones. This opportunity is appealing because it
drastically simplify the deployment of large distributed applications
even with complex network settings such as firewall, proxies or Net
Address Translation (NAT).

%Some attractive applications such as Firefox Hello~\cite{firefoxhello}
%or WebTorrent~\cite{webtorrent} demonstrate how video-conferencing or
%file-sharing directly within browsers becomes simple of access to
%end-users without third-party providers.

However, WebRTC has several constraints that makes existing
peer-sampling service inefficient or unreliable. First, browsers can
run on mobile devices and keeping the number of connections as low as
possible is a major requirement. Unfortunately, peer-sampling service
such as \CYCLON{}~\cite{ voulgaris2005cyclon} is not able to adapt the
number of connections to the real number of
participants. Consequently, a user on a mobile phone can maintain 10
connections with others remote browsers, even if just 3 was
enough. Adapting the number of connections to the number of
participants is crucial and is handled by peer-sampling service such
as \SCAMP{}~\cite{ganesh2003peer}. Unfortunately, browser-to-browser connection
in WebRTC requires a three-way handshake which is much more costly and
likely to fail compared to traditional settings. We show in this paper
that three-way handshake makes SCAMP fail. 

% PASCAL: too soon, keep that for Pb statement
% Knowingly, the challenge is about
% \begin{inparaenum}[(i)]
% \item keeping an optimal $\Theta(\ln |\mathcal{N}|)$ on the number of
%   connections alive\cite{erdos1959random} (where $\mathcal{N}$ is the set of
%   network members),
% \item keeping an optimal $O(1)$ complexity on connection establishments,
% \item keeping an optimal $\Theta(\exp \, t^{-1})$ convergence speed to a random
%   graph (where $t$ is an arbitrary time frame).
% \end{inparaenum}

In this paper, we introduce \SCAMPLON{}, a random peer sampling
protocol inspired by both \SCAMP{}~\cite{ganesh2003peer} and
\CYCLON{}~\cite{voulgaris2005cyclon}. Compared to state of art,
\begin{inparaenum}[(i)]
\item It dynamically adapts the neighborhood of each peer. Thus, the number of
  connections grows logarithmically compared to the size of the network.
\item It only uses neighbor-to-neighbor interactions to establish the
  connections. Thus, the connections are established in constant time.
\item It converges exponentially fast to a random graph. Thus, the network
  becomes quickly resilient to failure, can broadcast messages
  efficiently etc.
\item In experimental setup, we demonstrate how adaptivity of
  \SCAMPLON{} saves connections compared to\CYCLON{} at the cost of a
  little overhead.
\end{inparaenum}


% Knowingly, it improves state-of-the-art approaches in the traditional
% connection set-up. This improvement is crucial in the three-way conn\ection
% set-up and opens the gate to a vast variety of distributed applications.


The rest of this paper is organized as follow: Section~\ref{sec:background}
introduces the necessary background to understand \SCAMPLON{} and highlights
our motivations. Section~\ref{sec:proposal} details the \SCAMPLON{} protocol.
Section~\ref{sec:experiments} shows the properties of \SCAMPLON{} and compare
them to state-of-the-art random peer sampling
approaches. Section~\ref{sec:relatedwork} reviews the related work. We conclude
and discuss about perspective in Section~\ref{sec:conclusion}.

%%% Local Variables:
%%% mode: latex
%%% TeX-master: "../paper"
%%% End:
