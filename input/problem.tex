
\section{Problem statement}
\label{sec:problem}

We consider a set $V$ of peers labeled from $p_1$ to $p_N$.  The peers can
crash or leave at any time without giving notice, and they can recover or
re-enter the network freely. We consider only non-byzantine peers.

A multiset of arcs $E = (p_f,\,p_t) \in V \times V$ allows $p_f$ to
communicate with $p_t$ but not the converse. The multiset of arcs is
dynamic, i.e., arcs appear and disappear over time. By $\mathcal{N}^t
\in V \times E$, we denote the state of the \emph{network overlay} at
time $t$.

The multiset of arcs starting from $p_f$ is called the \emph{partial
  view} of $p_f$, noted $\mathcal{P}_f$. The destination peers in this
view are the \emph{neighbors} of $p_f$. A partial view can contain
arcs that are stale when neighbors have left or crashed.

The peers communicate by means of messages. They only send messages to
their neighbors. A departed peer cannot send messages. An arc may fail
to deliver a message. A stale arc systematically fails.

Building an adaptive-by-design random peer-sampling that meets WebRTC
constraints raises the following scientific problem:
\begin{problem}
  Let $t$ be an arbitrary time frame, let $V^t$ be the network membership at
  that given time $t$ and let $\mathcal{P}_x^t$ be the partial view of peer
  $p_x \in \mathcal{V}^t$.  An adaptive and reliable random peer-sampling should
  provide the following properties:
  \begin{enumerate}
  \item Partial view size: \hfill
    $\forall p_x \in V^t,\, |\mathcal{P}_x^t| = \mathcal{O} (\ln
    |V^t|)$      
  \item Connection establishment by a node: \hfill $\mathcal{O}(1)$
    %% \item  \begin{center}
    %%     Convergence speed: \hfill $\Theta(\exp \, t^{-1})$
    %%   \end{center}
  \end{enumerate}
\end{problem}

The first condition states that the partial view size is relative to the size of
the network at any time. It also states that partial views should grow and
shrink, at worst, logarithmically compared to the size of the network. This is a
condition for adaptiveness and not a requirement for network connectivity. The
second condition states that a node should be able to establish a connection by
relying only on a constant number of intermediary peers, independently of
network size.
%% The last condition states that the network
%% overlay converges exponentially fast to a topology that is close to a random
%% graph.
Lpbcast, HyParView, Newscast, and \CYCLON fail to meet the first
condition of the problem statement since they do not adapt their views
to the network size. \SCAMP fails to meet the second condition of the
problem statement since each connection implies an unsafe random
dissemination. The next session introduces \SPRAY, a protocol that
meets both of these constraints without any global knowledge.


%%% Local Variables:
%%% mode: latex
%%% TeX-master: "../paper"
%%% End:
