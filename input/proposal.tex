
\section{Scamplon}
\label{sec:proposal}

Scamplon\footnote{Scamplon stands as the contraction of Scamp and Cyclon.} is a
random peer sampling protocol inspired by both Scamp and Cyclon. It provides
the best of its parents: a logarithmically increasing partial view size
compared to the global network size, and fast convergence to a random graph
using only neighbour-to-neighbour connection establishments. As such, it
constitutes an improvement over state-of-the-art approaches in the common
context of one-way connections. Additionally, it greatly outperforms
state-of-the-art in the three-way handshake connection establishments. The
latter context becomes increasingly important with the apparition of technology
allowing peer-to-peer within modern web browsers. This section details the
Scamplon protocol through intuitions, examples and algorithms.

\begin{algorithm}
  
\small
\algrenewcommand{\algorithmiccomment}[1]{\hskip2em$\rhd$ #1}

\newcommand{\comment}[1]{$\rhd$ #1}


\algblockdefx[initially]{initially}{endInitially}
  [0] {\textbf{INITIALLY:}} 

\algblockdefx[act]{act}{endAct}
  [0] {\textbf{ACTIVE THREAD:}}

\algblockdefx[pas]{pas}{endPas}
  [0] {\textbf{PASSIVE THREAD:}}


\newcommand{\LINEFOR}[2]{%
  \algorithmicfor\ {#1}\ \algorithmicdo\ {#2} %
  }

\newcommand{\LINEIFTHEN}[2]{%
  \algorithmicif\ {#1}\ \algorithmicthen\ {#2} %
  }

\newcommand{\INDSTATE}[1][1]{\State\hspace{\algorithmicindent}}

\begin{algorithmic}[1]
  \Statex
  \initially
    \State $\mathcal{P} \leftarrow [\,]$;
    \hfill \comment{the partial view, sorted by age}
    \State $p$ ; \hfill \comment{identity of the local peer}
    \State $c \leftarrow 2$;
    \hfill \comment{additionnal connections when joining}

  \endInitially
  
  \act
    \Function{loop}{ } \hfill \comment{Every $\Delta$ time}
    \State $\mathcal{P} \leftarrow incrementAge(\mathcal{P})$;
    \State \textbf{let} $q \leftarrow getOldest(\mathcal{P})$;
    \State \textbf{let} $sample \leftarrow getSample(\mathcal{P}\setminus\left\{q\right\},\, \left \lceil{|\mathcal{P}|\over{2}} \right \rceil) \cup \left\{\langle p,\, 0 \rangle\right\}$;
    \State $sendTo(q,\, 'exchange',\, sample)$;
    \State \textbf{let} $sample'\leftarrow receiveFrom(q)$;
    \State $sample' \leftarrow replace(sample',\, p,\, q)$;
    \State $\mathcal{P} \leftarrow ((\mathcal{P} \setminus sample) \setminus \left\{ q \right\}) \cup sample'$;  \hfill \comment{update $\mathcal{P}$}
    \EndFunction
  \endAct
  
  \pas
    \Function{onExchange}{$o,\, sample$} \hfill \comment{$o: origin$}
    \State \textbf{let} $sample' \leftarrow getSample(\mathcal{P} ,\, \left\lceil |\mathcal{P}|\over{2} \right\rceil )$;
    \State $sendTo(o ,\, sample')$;
    \State $\mathcal{P} \leftarrow ((\mathcal{P} \setminus sample') \setminus \left\{ o \right\}) \cup sample$; \hfill \comment{update $\mathcal{P}$}
    \EndFunction
    \Statex
    \Function{onContact}{$o$} \hfill \comment{$o: origin$}
    \State \LINEFOR{\textbf{each} $q\in\mathcal{P}$}
    {$sendTo(q,\, 'fwdSubs',\, o)$;}
    \State \LINEFOR{$i \leftarrow 0$ \textbf{to} $c$}
    {\INDSTATE $sendTo(\mathcal{P}[\left\lfloor rand()*
        |\mathcal{P}|\right\rfloor],\, 'fwdSubs',\, o)$;}
    \EndFunction
    \Statex
    \Function{onFwdSubs}{$o$} \hfill \comment{$o: origin$}
    \If {$((rand() < (1 / (|\mathcal{P}|+1))) \wedge (o\not\in\mathcal{P}) 
      \wedge (o\neq p))$}
    \State $\mathcal{P} \leftarrow
    \mathcal{P}\cup \left\{\langle o,\, 0 \rangle\right\}$;
    \Else
    \State $sendTo(\mathcal{P}[\left\lfloor rand()*
      |\mathcal{P}|\right\rfloor],\, 'fwdSubs',\, o)$;
    \EndIf
    \EndFunction
  \endPas
  
\end{algorithmic}

  \caption{\label{algo:scamplon}The Scamplon protocol.}
\end{algorithm}

%%% Local Variables:
%%% mode: latex
%%% TeX-master: "../paper"
%%% End:
