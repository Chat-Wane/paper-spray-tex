
\section{\SCAMPLON{}}
\label{sec:proposal}

\SCAMPLON{}\footnote{\SCAMPLON{} stands as the contraction of \SCAMP{} and
  \CYCLON{}.} is a random peer sampling protocol inspired by both \SCAMP{} and
\CYCLON{}. It provides the best of its parents: 
\begin{inparaenum}[(i)]
\item a logarithmically increasing partial view size compared to the global
  network size,
\item and fast convergence to a random graph
\item using only neighbour-to-neighbour connection establishments.
\end{inparaenum}
It improves the state-of-the-art approaches in the traditionnal context of
one-way connections and greatly outperforms the
state-of-the-art~\cite{ganesh2001scamp,voulgaris2005cyclon} in the context
three-way handshake connection set up.  The latter becomes increasingly
important with the appearance of technologies allowing peer-to-peer within
modern web browsers. This section details the \SCAMPLON{} protocol through
intuitions, examples and algorithms.

\begin{asparadesc}
\item [Scamplon] reuses the joining process of \SCAMP{} to incrementally build
  the network. While such network is not flawless, it has the interesting
  property of logarithmically increasing the number of connections at each
  join, compared to the global network size. Still, newest members have a small
  partial view size, and oldest peers are more clustered. To alleviates these
  issues, a periodic \CYCLON{}-like protocol takes place in order to balance
  the partial views, using only neighbour-to-neighbour interactions. This
  balancing concerns both the partial view sizes, and the uniformity of chosen
  peers within them. Unfortunately, \CYCLON{} handles fixed-size partial
  views. Therefore, it must be adapted to handle partial views that grow and
  shrink as the network dynamically adapts to the membership.
\end{asparadesc}

The first issue of \CYCLON{} concerns the partial view sizes. For the recall,
the user of \CYCLON{} must foresee the maximum size of the network to set the
partial view size, and the size of the subset to exchange from
neighbour-to-neighbour.  However, in \SCAMPLON{}, a peer with $2$ neighbours in
its partial view can exchange with a peer with $10$ neighbours. Ideally, the
resulting size of both partial views would be $6$ after the exchange. Yet,
reaching this ideal value in one cycle is difficult without the knowledge of
each other's partial view size. Instead of acquiring this knowledge,
\SCAMPLON{} aims to converge to the ideal value by averaging their view size
over exchanges. Therefore, the initiating peer sends
$\left\lceil|\mathcal{P}|\over{2}\right\rceil$ members to the chosen peer
(chosen by age). The latter sends back
$\left\lceil|\mathcal{P}|\over{2}\right\rceil$ members too. After the receipt,
both peers remove the sent members and add the received members. It worth
noting that the global number of connections after the process must not
change. Otherwise, \SCAMPLON{} cannot guarantee that the partial view sizes are
logarithmic in average compared to the network size.  A remarkable difference
regarding \CYCLON{} is that \SCAMPLON{} explicitly allows to have a same
neighbour multiple times in a partial view. It impacts on the clustering
coefficient. However, the impact is not significant since these artefacts are
not numerous because the graph is random, and because they disappear over the
\SCAMPLON{} cycles (EXAMPLIFY).

There exists a close relationship between \SCAMPLON{} and the proactive
aggregation protocol introduced
in~\cite{jelasity2004epidemic,montresor2004robust}. The latter states that,
under the assumption of a peer sampling sufficiently random, the mean value
$\mu$ and the variance $\sigma^2$ at a given cycle $i$ are:
\begin{center}
  $\mu_i = {1\over{|\mathcal{N}|}} \sum\limits_{x \in \mathcal{N}} a_{i,\,x}$
  \hfill
  $\sigma^2_i = {1\over{|\mathcal{N}|-1}}\sum\limits_{x \in \mathcal{N}}
  (a_{i,\,x} - \mu_i)^2$
\end{center}
where $a_{i,\,x}$ is the value held by Peer $p_x$ at cycle $i$. The estimated
variance must converge to $0$ over cycles. In the \SCAMPLON{} case, the value
$a_{i,\,x}$ is the partial view size of Peer $p_x$ at cycle $i$. Indeed, each
exchange from Peer $p_1$ to Peer $p_2$ is an aggregation resulting to:
$|\mathcal{P}_1|\approx|\mathcal{P}_2|\approx{|\mathcal{P}_1| + |\mathcal{P}_2|
  \over{2}}$.
Furthermore, each peer performs this exchange protocol $1+Poisson(1)$ times per
cycle (i.e. they initiate one and receive one in average).  This relation being
established, we are able to conclude that \SCAMPLON{} converges exponentially
fast. More precisely, each cycle decreases the variance estimation of the
overall system by ${2\sqrt{\text{e}}}$.


\begin{asparadesc}
\item [Algorithm]\ref{algo:scamplon} shows the \SCAMPLON{} protocol running at
  each peer. It is divided between an active thread looping to update the
  partial view, and the passive thread which provides the reactions to received
  messages. The functions which are not explicitly defined are the following:
  \begin{itemize}
  \item $incrementAge(view)$: increments the age of each elements in the view
    and returns the modified view.
  \item $getOldest(view)$: retrieves the oldest of peers contained in the view
  \item $getSample(view, \, size)$: returns a sample of the view containing
    $size$ elements.
  \item $replace(view,\,old,\,new)$: replaces in the view all occurrences of
    the $old$ element by the $new$ element and returns the modified view.
  \item $rand()$: generates a random number between 0 and 1.
  \end{itemize}
  A peer $o$ joining the network will reach a contact peer. The latter will
  call Function $onContact$ which spreads $(|\mathcal{P}|+c)$ copies of $o$
  inside the network. Then, each time a peer receives one of these message,
  assuming that this peer does not have $o$ already and this peer is not $o$,
  it has a probability $1\over{|\mathcal{P}|+1}$ to integrate $o$ in
  $\mathcal{P}$. Otherwise it forwards the copy to a random neighbour.  At this
  point, $o$ has $1$ neighbour in its partial view, and appears
  $(c+1)log(|\mathcal{N}|)$ times in partial views of other members. The active
  thread aims to balance the partial views. Each time Function $loop$ is
  called, the age of each element in $\mathcal{P}$ is incremented. Then, the
  oldest peer $q$ is chosen to exchange a subset of their partial view. If Peer
  $q$ cannot be reached (i.e. it crashed/left), it is removed from the partial
  view and the operation is repeated. Once the initiating peer $p$ found a
  reachable peer $q$, the former selects a sample of its partial view,
  excluding $q$ and including itself. The size of this sample is half of its
  partial view, with a minimum of one peer: the initiating peer. The answer of
  $q$ is of the exact same kind. Since peers can appear multiple times in
  $\mathcal{P}$, the exchanging peers may send references to the other peer,
  e.g., Peer $o$'s sample can contain references to $q$. Such sample, without
  further processing, would create self-loop ($q$'s partial view contains
  references to $q$). To alleviate this undesirable behaviour, all occurrences
  of the other peer are replaced with the emitting peer.  Afterwards, both of
  them remove the sent sample from their view, remove the chosen neighbour, and
  add the received sample.
\end{asparadesc}

Note that extending the algorithm to provide handshaking is not difficult: it
only requires to keep track of the neighbour from where the membership messages
arrived, and forward the answer to this neighbour accordingly.

\begin{algorithm}
  
\small
\algrenewcommand{\algorithmiccomment}[1]{\hskip2em$\rhd$ #1}

\newcommand{\comment}[1]{$\rhd$ #1}


\algblockdefx[initially]{initially}{endInitially}
  [0] {\textbf{INITIALLY:}} 

\algblockdefx[act]{act}{endAct}
  [0] {\textbf{ACTIVE THREAD:}}

\algblockdefx[pas]{pas}{endPas}
  [0] {\textbf{PASSIVE THREAD:}}


\newcommand{\LINEFOR}[2]{%
  \algorithmicfor\ {#1}\ \algorithmicdo\ {#2} %
  }

\newcommand{\LINEIFTHEN}[2]{%
  \algorithmicif\ {#1}\ \algorithmicthen\ {#2} %
  }

\newcommand{\INDSTATE}[1][1]{\State\hspace{\algorithmicindent}}

\begin{algorithmic}[1]
  \Statex
  \initially
    \State $\mathcal{P} \leftarrow [\,]$;
    \hfill \comment{the partial view, sorted by age}
    \State $p$ ; \hfill \comment{identity of the local peer}
    \State $c \leftarrow 2$;
    \hfill \comment{additionnal connections when joining}

  \endInitially
  
  \act
    \Function{loop}{ } \hfill \comment{Every $\Delta$ time}
    \State $\mathcal{P} \leftarrow incrementAge(\mathcal{P})$;
    \State \textbf{let} $q \leftarrow getOldest(\mathcal{P})$;
    \State \textbf{let} $sample \leftarrow getSample(\mathcal{P}\setminus\left\{q\right\},\, \left \lceil{|\mathcal{P}|\over{2}} \right \rceil) \cup \left\{\langle p,\, 0 \rangle\right\}$;
    \State $sendTo(q,\, 'exchange',\, sample)$;
    \State \textbf{let} $sample'\leftarrow receiveFrom(q)$;
    \State $sample' \leftarrow replace(sample',\, p,\, q)$;
    \State $\mathcal{P} \leftarrow ((\mathcal{P} \setminus sample) \setminus \left\{ q \right\}) \cup sample'$;  \hfill \comment{update $\mathcal{P}$}
    \EndFunction
  \endAct
  
  \pas
    \Function{onExchange}{$o,\, sample$} \hfill \comment{$o: origin$}
    \State \textbf{let} $sample' \leftarrow getSample(\mathcal{P} ,\, \left\lceil |\mathcal{P}|\over{2} \right\rceil )$;
    \State $sendTo(o ,\, sample')$;
    \State $\mathcal{P} \leftarrow ((\mathcal{P} \setminus sample') \setminus \left\{ o \right\}) \cup sample$; \hfill \comment{update $\mathcal{P}$}
    \EndFunction
    \Statex
    \Function{onContact}{$o$} \hfill \comment{$o: origin$}
    \State \LINEFOR{\textbf{each} $q\in\mathcal{P}$}
    {$sendTo(q,\, 'fwdSubs',\, o)$;}
    \State \LINEFOR{$i \leftarrow 0$ \textbf{to} $c$}
    {\INDSTATE $sendTo(\mathcal{P}[\left\lfloor rand()*
        |\mathcal{P}|\right\rfloor],\, 'fwdSubs',\, o)$;}
    \EndFunction
    \Statex
    \Function{onFwdSubs}{$o$} \hfill \comment{$o: origin$}
    \If {$((rand() < (1 / (|\mathcal{P}|+1))) \wedge (o\not\in\mathcal{P}) 
      \wedge (o\neq p))$}
    \State $\mathcal{P} \leftarrow
    \mathcal{P}\cup \left\{\langle o,\, 0 \rangle\right\}$;
    \Else
    \State $sendTo(\mathcal{P}[\left\lfloor rand()*
      |\mathcal{P}|\right\rfloor],\, 'fwdSubs',\, o)$;
    \EndIf
    \EndFunction
  \endPas
  
\end{algorithmic}

  \caption{\label{algo:scamplon}The \SCAMPLON{} protocol.}
\end{algorithm}

There are few optimisations concerning the establishments of connections. For
instance, when a peer $p$ starts an exchange with $q$, and $q$ has $p$ in its
partial view, instead of inverting the link between $p$ and $q$, and $q$ and
$p$, \SCAMPLON{} does not change them. Another optimisation concerns a peer
having a neighbour multiple times in its partial view. While \SCAMPLON{} keeps
such information in its partial view, only one connection per neighbour is
necessary.


%%% Local Variables:
%%% mode: latex
%%% TeX-master: "../paper"
%%% End:
