
\section{Spray}
\label{sec:proposal}

\SPRAY is an adaptive random peer sampling protocol inspired by
\SCAMP~\cite{ganesh2003peer} and \CYCLON~\cite{voulgaris2005cyclon}. It
comprises three parts representing the lifecycle of a peer in the network.
Firstly, the joining process injects a logarithmically growing number of arcs
in the network. Hence, the number of arcs scales with the network size.  Next,
each peer runs a periodic process in order to balance the partial views both in
terms of partial view size, and uniformity of the referenced peers within
them. Quickly, the overlay network converges to a topology exposing properties
similar to random graphs. Finally, A peer is able to leave at any time without
giving notice (equivalent to a crash), still the network properties do not
degrade.

The key of adaptiveness consists in keeping a constant global number of arcs
during the whole protocol. Indeed, unlike \CYCLON, \SPRAY is always on the edge
of the optimal number of arcs compared to the network size. Since \SPRAY never
creates additional arcs after the joining, any removal is a definitive
loss. Thus, firstly, \SPRAY's joining adds some arcs to the network. Secondly,
\SPRAY's shuffling preserves all arcs in the network. Thirdly, \SPRAY's leaving
cautiously removes some arcs, ideally the number of arcs introduced by the last
joining peer.

Keeping global number of arcs constant can force, in some situations,
shuffling and leaving process to inject \emph{duplicates} in partial
views i.e. a partial view can contain multiple occurrences of a
particular neighbor. In this paper, we show that the number of
duplicates remains low and do not impact the network connectedness.

% This requirement introduces the main difference of \SPRAY compared to the
% state-of-art approaches. It uses a multiset instead of a set. In \SPRAY, the
% multisets can have different sizes. Yet, they very rapidly converge to a
% roughly identical size. Also, the multisets can contain multiple occurrences of
% a particular neighbor. Yet, these duplicates remain in small number and do not
% endanger the network connectedness. This section shows how \SPRAY manages this
% type of partial view to adapt itself to the network size.

\subsection{Joining}

%%%%%%%%%%%%%%%%%%%%%%
%% moved to the related work so it appears at top of page 
%%%%%%%%%%%%%%%%%%%%%
% \begin{figure*}
%   \centering
%   \subfloat[Figure A][$p_2$ executes $onSubs(p_1$)]{
%     
\begin{tikzpicture}[scale=1.2]

  \newcommand\X{35pt};
  \newcommand\Y{15pt};

  \draw(-0.75*\X, 0pt); %% positioning
  \draw( 2.75*\X, 0pt); %% positioning

  \scriptsize
  \draw[->,dashed,very thick](5+0*\X, 0*\Y) -- 
  node[anchor=south]{(a)}(-5+ 2*\X, 0*\Y);
  \draw[->] (-5+2*\X, 5pt) -- (5+\X, \Y);
  \draw[->] (-5+2*\X, 5pt) --  (5+\X, 2*\Y);
  \draw[->] (-5+2*\X, -5pt) -- (5+\X, -\Y);
  \draw[->] (-5+2*\X, -5pt) -- (5+\X, -2*\Y);

  \draw[fill=white, very thick]
  (0*\X, 0*\Y) node{$p_1$} +(-5pt,-5pt) rectangle +(5pt,5pt);
  \draw[fill=white, very thick]
  (2*\X, 0*\Y) node{$p_2$} +(-5pt,-5pt) rectangle +(5pt,5pt);

  \draw[fill=white](1*\X,2*\Y) node{$p_6$} +(-5pt,-5pt) rectangle +(5pt,5pt);
  \draw[fill=white](1*\X,1*\Y) node{$p_5$} +(-5pt,-5pt) rectangle +(5pt,5pt);
  \draw[fill=white](1*\X,-1*\Y) node{$p_4$} +(-5pt,-5pt) rectangle +(5pt,5pt);
  \draw[fill=white](1*\X,-2*\Y) node{$p_3$} +(-5pt,-5pt) rectangle +(5pt,5pt);
  
\end{tikzpicture}}
%   \hspace{40pt}
%   \subfloat[Figure B][$p_2$ calls $fwdSubs(p_1)$ on its neighbor]{
%     
\begin{tikzpicture}[scale=1.2]

  \newcommand\X{35pt};
  \newcommand\Y{15pt};

  \scriptsize
  \draw[->](5+0*\X, 0*\Y) -- (-5+ 2*\X, 0*\Y);
  \draw[->, very thick] (-5+2*\X, 5pt) -- (5+\X, \Y);
  \draw[->, very thick] (-5+2*\X, 5pt) --
  node[anchor=south west]{(b)} (5+\X, 2*\Y);
  \draw[->, very thick] (-5+2*\X, -5pt) -- (5+\X, -\Y);
  \draw[->, very thick] (-5+2*\X, -5pt) --
  node[anchor=north west]{(b)}(5+\X, -2*\Y);

  \normalsize
  \draw[fill=white]
  (0*\X, 0*\Y) node{$p_1$} +(-5pt,-5pt) rectangle +(5pt,5pt);
  \draw[fill=white, very thick]
  (2*\X, 0*\Y) node{$p_2$} +(-5pt,-5pt) rectangle +(5pt,5pt);

  \draw[fill=white, very thick]
  (1*\X,2*\Y) node{$p_6$} +(-5pt,-5pt) rectangle +(5pt,5pt);
  \draw[fill=white, very thick]
  (1*\X,1*\Y) node{$p_5$} +(-5pt,-5pt) rectangle +(5pt,5pt);
  \draw[fill=white, very thick]
  (1*\X,-1*\Y) node{$p_4$} +(-5pt,-5pt) rectangle +(5pt,5pt);
  \draw[fill=white, very thick]
  (1*\X,-2*\Y) node{$p_3$} +(-5pt,-5pt) rectangle +(5pt,5pt);

\end{tikzpicture}}
%   \hspace{40pt}
%   \subfloat[Figure C][$p_{3-6}$ execute $fwdSubs(p_1)$]{
%     
\begin{tikzpicture}[scale=1.2]

  \newcommand\X{35pt};
  \newcommand\Y{15pt};

  \scriptsize
  \draw[->](5+0*\X, 0*\Y) -- (-5+ 2*\X, 0*\Y);
  \draw[->] (-5+2*\X, 5pt) -- (5+\X, \Y);
  \draw[->] (-5+2*\X, 5pt) -- (5+\X, 2*\Y);
  \draw[->] (-5+2*\X, -5pt) -- (5+\X, -\Y);
  \draw[->] (-5+2*\X, -5pt) -- (5+\X, -2*\Y);

  \draw[->,dashed, very thick](-5+\X, 2*\Y) --
  node[anchor=south east]{(c)} ( 5pt,5pt);
  \draw[->,dashed, very thick](-5+\X, 1*\Y) -- ( 5pt,5pt);
  \draw[->,dashed, very thick](-5+\X, -1*\Y) -- ( 5pt,-5pt);
  \draw[->,dashed, very thick](-5+\X, -2*\Y) --
  node[anchor=north east]{(c)}( 5pt,-5pt);

  \normalsize
  \draw[fill=white, very thick]
  (0*\X, 0*\Y) node{$p_1$} +(-5pt,-5pt) rectangle +(5pt,5pt);
  \draw[fill=white]
  (2*\X, 0*\Y) node{$p_2$} +(-5pt,-5pt) rectangle +(5pt,5pt);

  \draw[fill=white, very thick]
  (1*\X,2*\Y) node{$p_6$} +(-5pt,-5pt) rectangle +(5pt,5pt);
  \draw[fill=white, very thick]
  (1*\X,1*\Y) node{$p_5$} +(-5pt,-5pt) rectangle +(5pt,5pt);
  \draw[fill=white, very thick]
  (1*\X,-1*\Y) node{$p_4$} +(-5pt,-5pt) rectangle +(5pt,5pt);
  \draw[fill=white, very thick]
  (1*\X,-2*\Y) node{$p_3$} +(-5pt,-5pt) rectangle +(5pt,5pt);
 

\end{tikzpicture}}
%   \caption{\label{fig:joiningexample}Example of the \SPRAY{}'s joining
%     protocol.}
% \end{figure*}

The \SPRAY's joining algorithm is the sole part of the protocol where the
global number of arcs in the network can increase. To meet the first
requirement of the problem statement, this number of arcs must grow
logarithmically compared to the network size. As in \SCAMP, we assume that
peers meet this constraint. Therefore, each of them uses this knowledge to
propagate the identity of the joining peer. Algorithm~\ref{algo:joining}
describes the \SPRAY's joining protocol. Line~\ref{line:multicast} shows that
the contacted peer only multicasts the new identity to its
neighborhood. Afterwards, to limit the risk of connection failures, each
neighbor immediately adds the joining peer to their own neighborhood. This
fulfills the second condition of the problem statement.  In total, the number
of arcs in the network increases of $1+\ln(|\mathcal{N}|)$ using only
neighbor-to-neighbor interactions.

\begin{figure*}
  \centering
  \subfloat[Figure A]
  [Peer $p_6$ initiates the exchange with $p_1$ by sending to the
  latter the multiset $\{p_6,\,p_9\}$.]{
    
\begin{tikzpicture}[scale=1.2]

  \newcommand\X{35pt};
  \newcommand\Y{15pt};

  \draw[->](5+0*\X, 0*\Y) -- (-5+ 2*\X, 0*\Y); %% 1 -> 2
  \draw[->] (-5+2*\X, 5pt) -- (5+\X, \Y);
  \draw[->](2*\X,5pt) -- (5+1*\X, 2*\Y); %% 2 -> 6
  \draw[->] (-5+2*\X, -5pt) -- (5+\X, -\Y);
  \draw[->] (-5+2*\X, -5pt) -- (5+\X, -2*\Y);

  \draw[->,very thick](-5+\X,2*\Y) -- (0pt,5pt); %% 6 -> 1

  \draw[->](-5+\X, 1*\Y) -- ( 5pt,5pt);
  \draw[->](-5+\X, -1*\Y) -- ( 5pt,-5pt);
  \draw[->](-5+\X, -2*\Y) -- ( 5pt,-5pt);

  \draw[->](-5+\X, 5+2*\Y)to[out=120,in=30](0pt,5+2*\Y); %% 6 -> 7
  \draw[->](-5+\X, 5+2*\Y)to[out=120,in=30](-5-\Y ,5+2*\Y); %% 6 -> 8
  \draw[->](-5+\X, 5+2*\Y)to[out=120,in=30](-10-2*\Y,5+2*\Y); %% 6 -> 9

  \normalsize
  \draw[fill=white, very thick]
  (0*\X, 0*\Y) node{$p_1$} +(-5pt,-5pt) rectangle +(5pt,5pt);
  \draw[fill=white](2*\X, 0*\Y) node{$p_2$} +(-5pt,-5pt) rectangle +(5pt,5pt);

  \draw[fill=white,very thick]
  (1*\X,2*\Y) node{$\mathbf{p_6}$} +(-5pt,-5pt) rectangle +(5pt,5pt);
  \draw[fill=white](1*\X,1*\Y) node{$p_5$} +(-5pt,-5pt) rectangle +(5pt,5pt);
  \draw[fill=white](1*\X,-1*\Y) node{$p_4$} +(-5pt,-5pt) rectangle +(5pt,5pt);
  \draw[fill=white](1*\X,-2*\Y) node{$p_3$} +(-5pt,-5pt) rectangle +(5pt,5pt);

  \draw[fill=white]( 0*\X,2*\Y)
  node{$p_7$} +(-5pt,-5pt) rectangle +(5pt,5pt);
  \draw[fill=white](-5+-\Y,2*\Y)node{$p_8$} +(-5pt,-5pt) rectangle +(5pt,5pt);
  \draw[fill=white](-10+-2*\Y,2*\Y) node{$\mathbf{p_9}$} +(-5pt,-5pt) rectangle +(5pt,5pt);
  

\end{tikzpicture}}
  \hspace{10pt}
  \subfloat[Figure B][Peer $p_1$ receives the $p_6$'s message. 
  It sends back the multiset $\{p_2\}$ and adds $\{p_6,\,p_9\}$ to its 
  partial view.]{
    
\begin{tikzpicture}[scale=1.2]

  \newcommand\X{35pt};
  \newcommand\Y{15pt};

  \draw[->](5+0*\X, 0*\Y) -- (-5+ 2*\X, 0*\Y); %% 1 -> 2
  \draw[->] (-5+2*\X, 5pt) -- (5+\X, \Y);
  \draw[->](2*\X,5pt) -- (5+1*\X, 2*\Y); %% 2 -> 6
  \draw[->] (-5+2*\X, -5pt) -- (5+\X, -\Y);
  \draw[->] (-5+2*\X, -5pt) -- (5+\X, -2*\Y);

  \draw[->,dashed, very thick](0pt,5pt)--(-5+\X, 2*\Y); %% 1 -> 6

  \draw[->](-5+\X, 1*\Y) -- ( 5pt,5pt);
  \draw[->](-5+\X, -1*\Y) -- ( 5pt,-5pt);
  \draw[->](-5+\X, -2*\Y) -- ( 5pt,-5pt);

  \draw[->](-5+\X, 5+2*\Y)to[out=120,in=30](0pt,5+2*\Y); %% 6 -> 7
  \draw[->](-5+\X, 5+2*\Y)to[out=120,in=30](-5-\Y ,5+2*\Y); %% 6 -> 8
  
  \draw[->,dashed, very thick](-5pt,5pt)--(-10-2*\Y,-5+2*\Y); %% 1 -> 9

  \normalsize
  \draw[fill=white, very thick]
  (0*\X, 0*\Y) node{$p_1$} +(-5pt,-5pt) rectangle +(5pt,5pt);
  \draw[fill=white](2*\X, 0*\Y)
  node{$\mathbf{p_2}$} +(-5pt,-5pt) rectangle +(5pt,5pt);

  \draw[fill=white,very thick]
  (1*\X,2*\Y) node{$p_6$} +(-5pt,-5pt) rectangle +(5pt,5pt);
  \draw[fill=white](1*\X,1*\Y) node{$p_5$} +(-5pt,-5pt) rectangle +(5pt,5pt);
  \draw[fill=white](1*\X,-1*\Y) node{$p_4$} +(-5pt,-5pt) rectangle +(5pt,5pt);
  \draw[fill=white](1*\X,-2*\Y) node{$p_3$} +(-5pt,-5pt) rectangle +(5pt,5pt);

  \draw[fill=white]( 0*\X,2*\Y)
  node{$p_7$} +(-5pt,-5pt) rectangle +(5pt,5pt);
  \draw[fill=white](-5+-\Y,2*\Y)node{$p_8$} +(-5pt,-5pt) rectangle +(5pt,5pt);
  \draw[fill=white](-10+-2*\Y,2*\Y) node{$p_9$} +(-5pt,-5pt) rectangle +(5pt,5pt);
  

\end{tikzpicture}}
  \hspace{10pt}
  \subfloat[Figure C][Peer $p_6$ receives the $p_1$'s response, it
  adds $\{p_2\}$ to its partial view.]{
    
\begin{tikzpicture}[scale=1.2]

  \newcommand\X{35pt};
  \newcommand\Y{15pt};

  \draw[->] (-5+2*\X, 5pt) -- (5+\X, \Y);
  \draw[->,dashed, very thick]
  (5+\X, 2*\Y)to[out=-20,in=110](2*\X, 5pt); %% 6 -> 2
  \draw[->](2*\X,5pt)to[out=160,in=-70](5+1*\X, 2*\Y); %% 2 -> 6
  \draw[->] (-5+2*\X, -5pt) -- (5+\X, -\Y);
  \draw[->] (-5+2*\X, -5pt) -- (5+\X, -2*\Y);

  \draw[->](0pt,5pt)--(-5+\X, 2*\Y); %% 1 -> 6

  \draw[->](-5+\X, 1*\Y) -- ( 5pt,5pt);
  \draw[->](-5+\X, -1*\Y) -- ( 5pt,-5pt);
  \draw[->](-5+\X, -2*\Y) -- ( 5pt,-5pt);

  \draw[->](-5+\X, 5+2*\Y)to[out=120,in=30](0pt,5+2*\Y); %% 6 -> 7
  \draw[->](-5+\X, 5+2*\Y)to[out=120,in=30](-5-\Y ,5+2*\Y); %% 6 -> 8
  
  \draw[->](-5pt,5pt)--(-10-2*\Y,-5+2*\Y); %% 1 -> 9

  \normalsize
  \draw[fill=white]
  (0*\X, 0*\Y) node{$p_1$} +(-5pt,-5pt) rectangle +(5pt,5pt);
  \draw[fill=white](2*\X, 0*\Y) node{$p_2$} +(-5pt,-5pt) rectangle +(5pt,5pt);

  \draw[fill=white,very thick]
  (1*\X,2*\Y) node{$p_6$} +(-5pt,-5pt) rectangle +(5pt,5pt);
  \draw[fill=white](1*\X,1*\Y) node{$p_5$} +(-5pt,-5pt) rectangle +(5pt,5pt);
  \draw[fill=white](1*\X,-1*\Y) node{$p_4$} +(-5pt,-5pt) rectangle +(5pt,5pt);
  \draw[fill=white](1*\X,-2*\Y) node{$p_3$} +(-5pt,-5pt) rectangle +(5pt,5pt);

  \draw[fill=white]( 0*\X,2*\Y)
  node{$p_7$} +(-5pt,-5pt) rectangle +(5pt,5pt);
  \draw[fill=white](-5+-\Y,2*\Y)node{$p_8$} +(-5pt,-5pt) rectangle +(5pt,5pt);
  \draw[fill=white](-10+-2*\Y,2*\Y) node{$p_9$} +(-5pt,-5pt) rectangle +(5pt,5pt);
  

\end{tikzpicture}}
  \caption{\label{fig:cyclicexample}Example of the \SPRAY's shuffling
    protocol. }
\end{figure*}

\begin{algorithm}

\small
\SetKwProg{Function}{function}{}{}
\SetKwProg{INITIALLY}{INITIALLY}{}{}
\SetKwProg{EVENTS}{EVENTS}{}{}
\DontPrintSemicolon
\LinesNumbered

\INITIALLY {} {
  $\mathcal{P} \leftarrow \varnothing$ \Comment{the partial view is a multiset} \;
}

\EVENTS {} {
  \Function{onSubs($o$)} { % \hfill \comm{$o: origin$}
    \lFor{\textbf{each} $\langle q,\,\_\, \rangle \in\mathcal{P}$}
    {$sendTo(q,\, 'fwdSubs',\, o)$} \label{line:multicast}
  }

  \BlankLine

  \Function{onFwdSubs($o$)} {% \hfill \comm{$o: origin$}
    $\mathcal{P} \leftarrow \mathcal{P}\uplus \left\{\langle o,\, 0 \rangle\right\}$
  }

}

\caption{\label{algo:joining}The joining protocol of \SPRAY.}
\end{algorithm}

The partial view is a multiset of pairs $\langle n,\, age\rangle$
which associate to the neighbor $n$ the age $age$. The multiset allows
managing duplicates and $age$ plays the same role than in \CYCLON i.e.
accelerate convergence by shuffling with oldest neighbors first. The $onSubs$ event
is called each time a peer joins the network. $onSubs$ forwards the
identity of the joining peer to all neighbors, indifferently of the
age. The $onFwdSubs$ event is called when a peer receives such
forwarded subscription. It adds the peer as one of its neighbor with
an age set to $0$ meaning that it is a brand new reference.

Figure~\ref{fig:joiningexample} depicts a joining scenario. In this scenario,
Peer $p_1$ contacts $p_2$ to join the network composed of $\{p_2$, $p_3$,
$p_4$, $p_5$, $p_6\}$. For simplicity sake, the figure shows only the new arcs
and the neighborhood of $p_1$ and $p_2$. Peer $p_1$ directly adds $p_2$ in its
partial view. Peer $p_2$ forwards the identity of $p_1$ to its
neighborhood. Each of these neighbors adds $p_1$ in their partial view. In
total, \SPRAY establishes 5 connections and the network is connected.

Unfortunately, the partial views of the newest peers are clearly unbalanced and
violates the first condition of our problem statement. The periodic protocol
described in Section~\ref{subsec:cyclic} will re-balance the partial views.

\subsection{Shuffling}
\label{subsec:cyclic}

Unlike \CYCLON, \SPRAY shuffles partial views of different sizes. The shuffling
aims to balance the partial view sizes and to randomly mix the neighborhood
between the peers. Nevertheless, the global number of arcs in the network is
invariant.

In \SPRAY's shuffling protocol, the involved peers send half their partial view
to each other. After integration, they both tend to the average of their
partial views and the sum of their partial views stays unchanged. In order to keep number of arcs invariant, \SPRAY uses a multiset to store partial views. If a peer receives a reference it already knows, it stores it as a duplicate to by incrementing the corresponding multiset entry. Thus,
the \SPRAY's shuffling protocol never increases nor decreases the arcs count.

%partial view using a set structure may lose information in the
%operation. Indeed, if one of the involved peer gives a reference to the other
%peer that the latter already know, one of the reference will overwrite the
%other. As consequence, one arc would be permanently lost. Over time, it would
%lead to a degeneration of the network, and eventually to partitions. \SPRAY
%tackles this issue by using a multiset. Using this structure, the peers always
%keep the received references, and always withdraw the given references. 

If duplicates have negative impact on the network properties, most of them will disappear after shuffling and proportionally become negligible as the network grows (see Section~\ref{subsec:duplicates}). 


\begin{algorithm}[h]
  
\small
\algrenewcommand{\algorithmiccomment}[1]{\hskip2em$\rhd$ #1}

\newcommand{\comm}[1]{$\rhd$ #1}

\algblockdefx[act]{act}{endAct}
  [0] {\textbf{ACTIVE THREAD:}}

\algsetblockdefx[pas]{pas}{endPas}
{65535}{}
[0] {\textbf{PASSIVE THREAD:}}


\newcommand{\LINEFOR}[2]{%
  \algorithmicfor\ {#1}\ \algorithmicdo\ {#2} %
  }

\newcommand{\LINEIFTHEN}[2]{%
  \algorithmicif\ {#1}\ \algorithmicthen\ {#2} %
  }

\newcommand{\INDSTATE}[1][1]{\State\hspace{\algorithmicindent}}

\begin{algorithmic}[1]
  \Statex
  \act
    \Function{loop}{ } \hfill \comm{Every $\Delta\,t$}
    \State $\mathcal{P} \leftarrow incrementAge(\mathcal{P})$;
    \State \textbf{let} $ \langle q,\, age \rangle \leftarrow getOldest(\mathcal{P})$;
    \State \textbf{let} $sample \leftarrow $ \label{line:samplesize}
    \Statex \hfill $getSample(\mathcal{P}\setminus\left\{\langle q, age\rangle\right\}, \left \lceil{|\mathcal{P}|\over{2}} \right \rceil-1) \uplus \left\{\langle p, 0 \rangle\right\}$;
    \State $sample \leftarrow replace(sample,\,q,\,p)$; \label{line:replace1}
    \State $sendTo(q,\, 'exchange',\, sample)$;
    \State \textbf{let} $sample'\leftarrow receiveFrom(q)$;
    \State $sample \leftarrow replace(sample,\,p,\,q)$;
    \State $\mathcal{P} \leftarrow (\mathcal{P} \setminus sample) \uplus
    sample'$;
    \EndFunction
  \endAct
  
  \pas
    \Function{onExchange}{$o,\, sample$} \hfill \comm{$o: origin$}
    \State \textbf{let} $sample' \leftarrow getSample(\mathcal{P} ,\, \left\lceil |\mathcal{P}|\over{2} \right\rceil )$;
    \State $sample' \leftarrow replace(sample',\,o,\,p);$ \label{line:replace2}
    \State $sendTo(o ,\, sample')$;
    \State $sample' \leftarrow replace(sample',\,p,\,o)$;
    \State $\mathcal{P} \leftarrow (\mathcal{P} \setminus sample') \uplus
    sample$; 
    \EndFunction
%%  \endPas
  
\end{algorithmic}

  \caption{\label{algo:scamplon}The cyclic protocol of \SPRAY.}
\end{algorithm}

Algorithm~\ref{algo:scamplon} shows the \SPRAY protocol running at each
peer. It is divided between an active thread looping to update the partial
view, and a passive thread which reacts to an exchange message. The functions
which are not explicitly defined are the following:
\begin{compactitem}
\item $incrementAge(view)$: increments the age of each elements in the view
  and returns the modified view.
\item $getOldest(view)$: retrieves the oldest of peers contained in the view.
\item $getSample(view, \, size)$: returns a sample of the view containing
  $size$ elements.
\item $replace(view,\,old,\,new)$: replaces in the view all occurrences of
  the $old$ element by the $new$ element and returns the modified view.
\item $rand()$: generates a random floating number between $0$ and $1$.
\end{compactitem}

In the active thread, Function $loop$ is called every $\Delta$ time
$t$. Firstly, the function increments the age of each neighbor in
$\mathcal{P}$. Then, the oldest peer $q$ is chosen to exchange a subset of its
partial view. If Peer $q$ cannot be reached (i.e. it crashed/left), the peer
$p$ executes the crash handling function (cf. Section~\ref{subsec:leaving}) and
repeats the process until it finds a reachable peer $q$. Thus, the aging
process (which is an inheritance from \CYCLON) speeds up the removal of
crashed/departed peers. Once, it finds a reachable neighbor $q$, Peer $p$
selects a sample of its partial view, excluding one occurrence of $q$ and
including itself. The size of this sample is half of its partial view, with at
least one peer: the initiating peer (cf. Line~\ref{line:samplesize}). The
answer of $q$ contains half of its partial view too. Since peers can appear
multiple times in $\mathcal{P}$, the exchanging peers may send references to
the other peer, e.g., Peer $o$'s sample can contain references to $q$. Such
sample, without further processing, would create self-loop ($q$'s partial view
contains references to $q$). To alleviate this undesirable behavior, all
occurrences of the other peer are replaced with the emitting peer
(cf. Line~\ref{line:replace1},~\ref{line:replace2}).  Afterwards, both of them,
remove the sample they sent from their view and add the received sample.

Figure~\ref{fig:cyclicexample} depicts the \SPRAY's cyclic
procedure. This scenario follows from Figure~\ref{fig:joiningexample}:
Peer $p_1$ just joined the network. Peer $p_6$ initiates an exchange
with $p_1$ (the oldest among the $p_6$'s partial view). It randomly
chooses $\left\lceil{|\mathcal{P}_6|\div 2}\right\rceil = 1$ peer
among its neighborhood. In this case, it picks $p_9$ from
$\{p_9,\,p_8,\,p_7\}$.  It sends the chosen peer plus its own identity
to Peer $p_1$. In response, the latter picks
$\left\lceil{|\mathcal{P}_1|\div 2}\right\rceil = 1$ peer from its
partial view. It sends back its sole neighbor $p_2$ and directly adds
the received neighbor to its partial view. After receipt, Peer $p_6$
removes the sent neighbors from its partial view, removes an
occurrence of $p_1$, and adds the received peer from $p_1$. The peers
$\{p_6,\,p_9\}$ compose the $p_1$'s partial view. The peers
$\{p_2,\,p_7,\,p_8\}$ compose the $p_6$'s partial view.

The example shows that, at first, the initiating peer has $4$ peers in
its partial view, while the receiving peer has only $1$ peer. Then,
after the exchange, the former has $3$ neighbors including $1$ new
peer. The receiving peer has $2$ neighbors, and both of them are
new. Thus, the periodic procedure tends to even up the partial view
size of network members. It also scatters neighbors in order to remove
the highly clustered groups which may appear because of the joining
protocol.

%% figure related to crash/departure, here to be on top of page
\begin{figure*}
  \centering
  \subfloat[Figure A][Peer $p_1$ crashes.]{
    
\begin{tikzpicture}[scale=1.2]

  \newcommand\X{35pt};
  \newcommand\Y{15pt};
  \large
  \draw[->](-5+\X, 1*\Y) --node{$\times$} ( 5pt,5pt);
  \draw[->](-5+\X, -1*\Y) --node{$\times$} ( 5pt,-5pt);
  \draw[->](-5+\X, -2*\Y) --node{$\times$} ( 5pt,-5pt);

  \draw[->](-5pt,5pt)--node{$\times$}(-10-2*\Y,-5+2*\Y); %% 1 -> 9
  \draw[->](-5pt,5pt)--node{$\times$}(-5-1*\Y,-5+2*\Y); %% 1 ->8 
  \draw[->](-5pt,5pt)--node{$\times$}(0pt,-5+2*\Y); %% 1 -> 7
  \draw[->](-5pt,5pt)--node{$\times$}(-5+\X,-5+2*\Y); %% 1 -> 6
  \normalsize
  \draw[->](5+ 1*\X, 5+ 1*\Y)--(-5+2*\X, 2*\Y); %% 5 -> 14
  \draw[->](5+1*\X,  1*\Y)--(-5+2*\X, 1*\Y); %% 5 -> 13 
  
  \draw[->](5+\X, 5-\Y) -- (-5+2*\X,0pt); %% 4 -> 12
  \draw[->](5+\X, -\Y) -- (-5+2*\X, -\Y); %% 4 -> 11
  
  \draw[->](5+\X, -2*\Y) -- (-5+2*\X, -2*\Y);
  
  \small
  \draw[fill=white,very thick]
  (0*\X, 0*\Y) node{$p_1$} +(-5pt,-5pt) rectangle +(5pt,5pt);
  \draw[thick] (-5pt,-5pt) -- (5pt,5pt);
  \draw[thick] (-5pt, 5pt) -- (5pt,-5pt);
  
  \draw[fill=white]
  (1*\X,1*\Y) node{$p_5$} +(-5pt,-5pt) rectangle +(5pt,5pt);
  \draw[fill=white]
  (1*\X,-1*\Y) node{$p_4$} +(-5pt,-5pt) rectangle +(5pt,5pt);
  \draw[fill=white]
  (1*\X,-2*\Y) node{$p_3$} +(-5pt,-5pt) rectangle +(5pt,5pt);

  \draw[fill=white](\X,2*\Y) node{$p_6$} +(-5pt,-5pt) rectangle +(5pt,5pt);

  \draw[fill=white]( 0*\X,2*\Y)
  node{$p_7$} +(-5pt,-5pt) rectangle +(5pt,5pt);
  \draw[fill=white](-5+-\Y,2*\Y)node{$p_8$} +(-5pt,-5pt) rectangle +(5pt,5pt);
  \draw[fill=white](-10+-2*\Y,2*\Y) node{$p_9$} +(-5pt,-5pt) rectangle +(5pt,5pt);
  
  \draw[fill=white](2*\X,2*\Y)node{$p_{14}$} +(-5pt,-5pt) rectangle +(5pt,5pt);
  \draw[fill=white](2*\X,1*\Y)node{$p_{13}$} +(-5pt,-5pt) rectangle +(5pt,5pt);
  \draw[fill=white](2*\X,0*\Y)node{$p_{12}$} +(-5pt,-5pt) rectangle +(5pt,5pt);
  \draw[fill=white](2*\X,-1*\Y)node{$p_{11}$}+(-5pt,-5pt) rectangle +(5pt,5pt);
  \draw[fill=white](2*\X,-2*\Y)node{$p_{10}$}+(-5pt,-5pt) rectangle +(5pt,5pt);

\end{tikzpicture}}
  \hspace{10pt}
  \subfloat[Figure B][The peers $p_{3-5}$ notice that they cannot 
  reach $p_1$ anymore.]{
    
\begin{tikzpicture}[scale=1.2]

  \newcommand\X{35pt};
  \newcommand\Y{15pt};

  \large
  \draw[->, very thick](-5+\X, 1*\Y) -- node{$\times$} ( 5pt,5pt);
  \draw[->, very thick](-5+\X, -1*\Y) --node{$\times$} ( 5pt,-5pt);
  \draw[->, very thick](-5+\X, -2*\Y) --node{$\times$} ( 5pt,-5pt);

  \normalsize

  \draw[->](5+ 1*\X, 5+ 1*\Y)--(-5+2*\X, 2*\Y); %% 5 -> 14
  \draw[->](  5+1*\X, 1*\Y)--(-5+2*\X, 1*\Y); %% 5 -> 13 (v)
  
  \draw[->](5+\X, 5-\Y) -- (-5+2*\X,0pt); %% 4 -> 12
  \draw[->](5+\X, -\Y) -- (-5+2*\X, -\Y); %% 4 -> 11
  
  \draw[->](5+\X, -2*\Y) -- (-5+2*\X, -2*\Y);
  
  \small
  \draw[fill=white]
  (0*\X, 0*\Y) node{$p_1$} +(-5pt,-5pt) rectangle +(5pt,5pt);
  \draw (-5pt,-5pt) -- (5pt,5pt);
  \draw (-5pt, 5pt) -- (5pt,-5pt);
  
  \draw[fill=white, very thick]
  (1*\X,1*\Y) node{$p_5$} +(-5pt,-5pt) rectangle +(5pt,5pt);
  \draw[fill=white, very thick]
  (1*\X,-1*\Y) node{$p_4$} +(-5pt,-5pt) rectangle +(5pt,5pt);
  \draw[fill=white, very thick]
  (1*\X,-2*\Y) node{$p_3$} +(-5pt,-5pt) rectangle +(5pt,5pt);

  \draw[fill=white](\X,2*\Y) node{$p_6$} +(-5pt,-5pt) rectangle +(5pt,5pt);

  \draw[fill=white]( 0*\X,2*\Y)
  node{$p_7$} +(-5pt,-5pt) rectangle +(5pt,5pt);
  \draw[fill=white](-5+-\Y,2*\Y)node{$p_8$} +(-5pt,-5pt) rectangle +(5pt,5pt);
  \draw[fill=white](-10+-2*\Y,2*\Y) node{$p_9$} +(-5pt,-5pt) rectangle +(5pt,5pt);
  
  \draw[fill=white](2*\X,2*\Y)node{$p_{14}$} +(-5pt,-5pt) rectangle +(5pt,5pt);
  \draw[fill=white](2*\X,1*\Y)node{$p_{13}$} +(-5pt,-5pt) rectangle +(5pt,5pt);
  \draw[fill=white](2*\X,0*\Y)node{$p_{12}$} +(-5pt,-5pt) rectangle +(5pt,5pt);
  \draw[fill=white](2*\X,-1*\Y)node{$p_{11}$}+(-5pt,-5pt) rectangle +(5pt,5pt);
  \draw[fill=white](2*\X,-2*\Y)node{$p_{10}$}+(-5pt,-5pt) rectangle +(5pt,5pt);

\end{tikzpicture}}
  \hspace{10pt}
  \subfloat[Figure C][The peers $p_3$ and $p_5$ choose to establish
  a duplicate with one of their existing neighbor.]{
    
\begin{tikzpicture}[scale=1.2]

  \newcommand\X{35pt};
  \newcommand\Y{15pt};

  \draw[->](5+ 1*\X, 5+ 1*\Y)--(-5+2*\X, 2*\Y); %% 5 -> 14
  \draw[->](5+1*\X, 2.5+ 1*\Y)--(-5+2*\X, 2.5+ 1*\Y); %% 5 -> 13 (^)
  \draw[->,dashed, very thick]
  (  5+1*\X,-2.5+1*\Y)--(-5+2*\X,-2.5+1*\Y); %% 5 -> 13 (v)
  
  \draw[->](5+\X, 5-\Y) -- (-5+2*\X,0pt); %% 4 -> 12
  \draw[->](5+\X, -\Y) -- (-5+2*\X, -\Y); %% 4 -> 11
  
  \draw[->](5+\X, 2.5-2*\Y) -- (-5+2*\X, 2.5-2*\Y);
  \draw[->,dashed, very thick](5+\X, -2.5-2*\Y) -- (-5+2*\X , -2.5-2*\Y);
  
  \small
  \draw[fill=white]
  (0*\X, 0*\Y) node{$p_1$} +(-5pt,-5pt) rectangle +(5pt,5pt);
  \draw (-5pt,-5pt) -- (5pt,5pt);
  \draw (-5pt, 5pt) -- (5pt,-5pt);
  
  \draw[fill=white, very thick]
  (1*\X,1*\Y) node{$p_5$} +(-5pt,-5pt) rectangle +(5pt,5pt);
  \draw[fill=white, very thick]
  (1*\X,-1*\Y) node{$p_4$} +(-5pt,-5pt) rectangle +(5pt,5pt);
  \draw[fill=white, very thick]
  (1*\X,-2*\Y) node{$p_3$} +(-5pt,-5pt) rectangle +(5pt,5pt);

  \draw[fill=white](\X,2*\Y) node{$p_6$} +(-5pt,-5pt) rectangle +(5pt,5pt);

  \draw[fill=white]( 0*\X,2*\Y)
  node{$p_7$} +(-5pt,-5pt) rectangle +(5pt,5pt);
  \draw[fill=white](-5+-\Y,2*\Y)node{$p_8$} +(-5pt,-5pt) rectangle +(5pt,5pt);
  \draw[fill=white](-10+-2*\Y,2*\Y) node{$p_9$} +(-5pt,-5pt) rectangle +(5pt,5pt);
  
  \draw[fill=white](2*\X,2*\Y)node{$p_{14}$} +(-5pt,-5pt) rectangle +(5pt,5pt);
  \draw[fill=white](2*\X,1*\Y)node{$p_{13}$} +(-5pt,-5pt) rectangle +(5pt,5pt);
  \draw[fill=white](2*\X,0*\Y)node{$p_{12}$} +(-5pt,-5pt) rectangle +(5pt,5pt);
  \draw[fill=white](2*\X,-1*\Y)node{$p_{11}$}+(-5pt,-5pt) rectangle +(5pt,5pt);
  \draw[fill=white](2*\X,-2*\Y)node{$p_{10}$}+(-5pt,-5pt) rectangle +(5pt,5pt);

\end{tikzpicture}}
  \caption{\label{fig:crashexample}Example of \SPRAY's crash/leaving
    handler. }
\end{figure*}

Concerning convergence time of the shuffling algorithm, there exists a close
relationship between \SPRAY and the proactive aggregation protocol introduced
in~\cite{jelasity2004epidemic,montresor2004robust}. It states that, under the
assumption of a peer sampling sufficiently random, the mean value $\mu$ and the
variance $\sigma^2$ at a given cycle $i$ are:
\begin{center}
  $\mu_i = {1\over{|\mathcal{N}|}} \sum\limits_{x \in \mathcal{N}} a_{i,\,x}$
  \hfill
  $\sigma^2_i = {1\over{|\mathcal{N}|-1}}\sum\limits_{x \in \mathcal{N}}
  (a_{i,\,x} - \mu_i)^2$
\end{center}
where $a_{i,\,x}$ is the value held by Peer $p_x$ at cycle $i$. The estimated
variance must converge to $0$ over cycles. In other terms, the values tends to
be the same over cycles. In the \SPRAY case, the value $a_{i,\,x}$ is the
partial view size of Peer $p_x$ at cycle $i$. Indeed, each exchange from Peer
$p_1$ to Peer $p_2$ is an aggregation resulting to:
$|\mathcal{P}_1|\approx|\mathcal{P}_2|\approx{(|\mathcal{P}_1| +
  |\mathcal{P}_2|) \div 2}$.
Furthermore, at each cycle, each peer is involved in the exchange protocol at
least once (they initiate one), and in the best case 1+Poisson(1) (they
initiate one and, in average, each peer receives another one). This relation
being established, we know that \SPRAY's partial view sizes converge
exponentially fast to the global average size. Additionally, we know that each
cycle decreases their variance in overall system at a rate comprised between
${1\div 2}$ and $1\div ({2\sqrt{\text{e}}})$.

The shuffling algorithm provides adaptiveness at the cost of
duplicates. Averaging size of partial views over exchanges provides exponential
convergence speed as expected.

\subsection{Leaving and crashes}
\label{subsec:leaving}

Using \SPRAY, the peers are free to leave the network without giving notice. In
that regard, we do not consider the departures and the crashes differently.
Without any appropriate reaction, the network may collapse due to an over
zealous removal of arcs. Indeed, when a peer joins the network, it injects in
it $1+\ln(|\mathcal{N}|)$ arcs. Nevertheless, after few exchanges, the partial
view of the joining peer becomes populated with more neighbors. Then, if this
peer leaves, it removes $\ln(|\mathcal{N}|)$ arcs from its partial view, and
another $\ln(|\mathcal{N}|)$ arcs from peers which have this peer in their
partial view. Therefore, without any crash handler, we remove
$2\ln(|\mathcal{N}|)$ connections instead of $1+\ln(|\mathcal{N}|)$. To
alleviate this issue, each peer that detects a crash may reestablish a
connection with anyone in its neighborhood (which will spread in the network
over the exchanges). The probability of reestablishing a connection is
$1-{1\div{|\mathcal{P}|}}$. Since ${|\mathcal{P}|}\approx \ln(|\mathcal{N}|)$
peers have the crashed peer in their partial view, it is likely that all of
them will reestablish a connection, excepted one. Therefore, when a peer
leaves, it approximately removes the number of connections it injected when it
joined.

\begin{algorithm}[h]
  
\small
\SetKwProg{Function}{function}{}{}
\SetKwComment{tcp}{$\triangleright$~}{}
\DontPrintSemicolon
\LinesNumbered

\newcommand{\LET}[0]{\textbf{let}\xspace}
\newcommand{\FROM}[0]{\textup{\textbf{from}}\xspace}
\newcommand{\TO}[0]{\textup{\textbf{to}}\xspace}

\Function{\textup{onPeerDown ($q$)} \tcp*[f]{$q$: crashed/departed}} {
  \LET $occ \leftarrow 0$ \;

  \ForEach(\tcp*[f]{remove and count}) { $\langle n, age\rangle \in \mathcal{P}$ }  {
    \If {$n=q$} {
       $\mathcal{P} \leftarrow \mathcal{P}\setminus \{\langle n,\,age\rangle \}$ \;
       $occ \leftarrow occ + 1$ \;
    }
  }

  \For{$i$ \FROM $0$ \TO $occ$} {
    \tcp*[l]{probabilistically duplicates}

    \If{$\textup{rand( )}>{1\div{(|\mathcal{P}|+occ}})$} {
       \LET $\langle n,\,\_ \,\rangle \leftarrow
         \mathcal{P}[\left\lfloor \textup{rand( )}*|\mathcal{P}|\right\rfloor]$ \;
       $\mathcal{P} \leftarrow \mathcal{P} \uplus \left\{\langle n,\, 0\rangle\right\}$
    }
  }
}

\BlankLine

\Function{\textup{onArcDown($q$, $age$)} \tcp*[f]{$q$: arc arrival}} {
  $\mathcal{P} \leftarrow \mathcal{P}\setminus \{\langle q, age\rangle \}$ \;
  \tcp*[l]{systematically duplicates}
  \LET $\langle n, \_ \rangle \leftarrow
  \mathcal{P}[\left\lfloor \textup{rand( )}*|\mathcal{P}|\right\rfloor]$ \;
  $\mathcal{P} \leftarrow \mathcal{P} \uplus \left\{\langle n, 0\rangle\right\}$ \;

}

  \caption{\label{algo:unreachable}The crash/departure handler of \SPRAY.}
\end{algorithm}

Algorithm~\ref{algo:unreachable} shows the manner in which \SPRAY deals with
departures and crashes.  Function $onPeerDown$ shows the reaction of \SPRAY
when the peer $q$ is detected as crashed/departed. A first loop counts the
occurrences of this neighbor in the partial view, and removes all of them. The,
the second loop probabilistically duplicates the reference of a known peer. The
probability depends of the partial view size before the removals.

Figure~\ref{fig:crashexample} depicts the \SPRAY's crash/leaving handler. The
scenario follows from prior examples after few other exchanges. Peer $p_1$
leaves the network without giving notice. With it, $7$ connections are
down. Peers $p_3$, $p_4$, and $p_5$ have the crashed/left peer in their partial
view. Peer $p_5$ has $1-{1\div{|\mathcal{P}_5|}}={2\div{3}}$ chance to replace
the dead connections. In this case, it duplicates the connection to
$p_{13}$. Identically, $p_3$ and $p_4$ detect the crash/leaving and run the
appropriate operation. Only $p_3$ duplicates one of its connection. In total,
$5$ connections have been removed.

The example shows that some peers reestablish connections if they detect a dead
connection. The probability depends on the partial view size of each of these
peer. In average, one of these peer will likely remove the arc while the other
peers will duplicate one of their existing arcs. In this case, Peer $p_1$
injected $5$ connections when it joined. It removes $7-2 =5 $ connections when
it leaves. The global number of connections remains logarithmic compared to the
number of members in the network. Nevertheless, we can see that connectedness
is not entirely guaranteed (only with the high probability implied by random
graphs). Indeed, if Peer $p_1$ is the sole bridge between two clusters, adding
arcs is not enough to ensure connectedness.

Algorithm~\ref{algo:unreachable} also shows that \SPRAY distinguishes peer
crashes and arc crashes. Indeed, Function $onArcDown$ deals with connection
establishments failures. In this function, the failing arc is systematically
replaced with a duplicate. Therefore, the arc count stays invariant even in
presence of connection establishment failures. The distinction between the
functions $onPeerDown$ and $onArcDown$ is necessary because the former is
supposed to remove a small arc quantity over departures, contrarily to the
latter. Without this small removal, the global arc count would grow unbounded
with network turnover.

In WebRTC context, \SPRAY calls the $onArcDown$ function when a connection
establishement fails. \SPRAY calls the $onPeerDown$ function when the
connection was established once but the neighbor is not responding.

% To summarize, \SPRAY provides:
% \begin{inparaenum}[(i)]
% \item a logarithmically increasing partial view size compared to the global
%   network size,
% \item a connections establishement.
% %%\item an exponentially fast convergence to a random graph.
% \end{inparaenum}
% Providing both these properties, \SPRAY improves the state-of-the-art
% approaches in the traditional connection set-up. Furthermore, the improvement
% becomes crucial in the context of three-way handshake connection set-up.  The
% latter becomes increasingly important with the appearance of technologies
% allowing peer-to-peer within modern web browsers.  The next section aims to
% demonstrate experimentally the behavior of \SPRAY. In particular, it aims to
% highlight the aforementioned properties.


%%% Local Variables:
%%% mode: latex
%%% TeX-master: "../paper"
%%% End:
