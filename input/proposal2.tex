
\section{Use case: real-time editing}
\label{sec:use-case}

In order to demonstrate the performance of \SPRAY with a real scenario, we
introduce \CRATE\footnote{\url{https://github.com/Chat-Wane/CRATE}}, a real-time
editor that allows editing anytime and anywhere, whatever the number of
participants, without any third party~\cite{nedelec2016crate}. Compared to
trending Cloud-based editors such as Google Docs, it alleviates privacy,
scalability, and single point of failure issues while remaining easy to use.  It
can be used for small groups but also during events such as massive online
lectures, TV shows, conferences that gather larger groups. For instance, online
lecture sessions reach thousands of students. Transitions between small groups
and large groups is supported transparently thanks to \SPRAY. Distributed
real-time editing is a pertinent context for \SPRAY, for group sizes are not the
same depending on the document, group sizes change quickly over time, and any
participant broadcasts its changes to all other participants.

\CRATE builds a network of browsers where each browsers is able to communicate
with a logarithmically scaling number of browsers compared to the global network
size. Each change performed on documents transits through neighbors and reaches
all members in a scalable way~\cite{birman1999bimodal}. Unlike the
state-of-the-art~\cite{tolgyeski2009adaptive, voulgaris2005cyclon}, \SPRAY
allows the diffusion cost to adapt to the network size without any probing
mechanism or any central authority. Without intervention of the developpers or
the users, \SPRAY makes sure that small networks do not pay the price of large
networks.

\subsection{Operation}

To provide availability and responsiveness in documents, collaborative
editors consider multiple authors, each hosting a replica of their
shared document~\cite{saito2005optimistic}.  On updates, the local
replica is directly modified. Then, each update is propagated to all
the editing session where it is integrated. Strong eventual
consistency~\cite{bailis2013eventual} states that a system is correct
if and only if the replicas converge to an equivalent state when they
integrated the same updates. In other terms, the users read identical
documents.

\CRATE uses a distributed sequence data type~\cite{shapiro2011conflict} which
provides two commutative operations: the insertion and the removal of an
element. Commutativity allow users to avoid the difficult, time consuming, and
error-prone task of solving conflicts due to concurrent edits. However,
commutativity comes at the price of metadata attached to each character. These
metadata (referred to as \emph{identifiers}) are unique and immutable. The
structure representing an identifier is a list $[\ell_1.\ell_2\ldots\ell_k]$ the
size $k$ of which is determined during allocation. \CRATE uses an allocator
function~\cite{nedelec2013lseq} to keep these identifiers under acceptable
growth, i.e., a polylogarithmic upper bound on their space complexity compared
to the document size: $\mathcal{O}((\log |document|)^2)$.

% An identifier is a list $[\ell_1.\ell_2\ldots\ell_k]$ where $k$ is the
% identifier depth. A total order among identifiers allows retrieving the
% sequence of characters. To encode such total order, the structure uses a simple
% lexicographic order comparing
% \begin{inparaenum}[(i)]
% \item paths -- chosen by the allocator and the most important as it determines
%   the space complexity of the approach --
% \item globally unique identifiers -- required for uniqueness and necessary in
%   some concurrent cases.
% \end{inparaenum}
% We focus our description on paths.

% \begin{figure}
%   \centering
%   \begin{tikzpicture}[scale=1.1]

\newcommand\Y{-45}
\newcommand\ADDY{-8}

  \scriptsize
  \draw[dashed] (340pt,10pt) node[anchor=north west]{maximum arity}  -- (340pt,3*\Y pt);
  \draw[dashed] (390pt,0 pt)  --(330pt,0 pt);
  \draw[dashed] (390pt,\Y pt) -- (330pt,\Y pt);
  \draw[dashed] (390pt,2*\Y pt) -- (330pt,2*\Y pt);
  \draw[dashed] (390pt,3*\Y pt) -- (330pt,3*\Y pt);

  \draw (340pt,0.5*\Y pt)
  node[anchor=west, align=center]{$32$};
  \draw (340pt,1.5*\Y pt)
  node[anchor=west, align=center]{$64$};
  \draw (340pt,2.5*\Y pt)
  node[anchor=west, align=center]{$128$};

  \begin{scope}[shift={(90pt,0pt)}]
\begin{scope}[shift={(160pt,0pt)}]
  \draw[->,dashed](50pt,3*\Y + 10 pt)--node[anchor=north]{insertion order}
  (-50pt,3*\Y + 10 pt);
  %% node to node
  \scriptsize
  \draw[thick] (0pt,0pt) -- node[anchor=south east]{0} (-70pt,\Y pt);
  \draw[thick] (0pt,0pt) -- node[anchor=east]{1} (50pt, \Y pt); %% Y
  \draw[thick] (-70pt, \Y pt) -- node[anchor=north]{57} (30pt, 2 * \Y pt); %% T
  \draw[thick] (-70pt, \Y pt) -- node[anchor=north]{56} (10pt, 2 * \Y pt); %% R
  \draw[thick] (-70pt, \Y pt) -- node[anchor=north]{53}(-10pt, 2 * \Y pt);%% E
  \draw[thick] (-70pt, \Y pt) -- node[anchor=north]{48}(-30pt,2 * \Y pt); %% W
  \draw[thick] (-70pt, \Y pt) -- node[anchor=east]{44}(-50pt,2 * \Y pt); %% Q
  \draw[dashed, thick] (0pt,0pt) -- node[anchor=south west]{32} (70pt,\Y pt);

  %% node to element
  \draw[->] ( 50pt, \Y pt) -- ( 50pt, \ADDY + \Y pt); %% Y
  \draw[->] ( 30pt, 2* \Y pt) -- ( 30pt, \ADDY + 2 *\Y pt); %% T
  \draw[->] ( 10pt, 2 *\Y pt) -- ( 10pt, \ADDY + 2 *\Y pt); %% R
  \draw[->] (-10pt, 2 *\Y pt) -- (-10pt, \ADDY + 2 *\Y pt); %% E
  \draw[->] (-30pt, 2 *\Y pt) -- (-30pt, \ADDY + 2 *\Y pt); %% W
  \draw[->] (-50pt, 2 *\Y pt) -- (-50pt, \ADDY + 2 *\Y pt); %% Q

  %% element to desambiguator
  \draw[->,densely dashdotted]
  ( 50pt, \ADDY + \Y pt) -- ( 50pt,2.75*\ADDY+\Y pt); %% Y
  \draw[->,densely dashdotted]
  ( 30pt, \ADDY + 2* \Y pt) -- ( 30pt,2.75*\ADDY+ 2* \Y pt); %% T
  \draw[->,densely dashdotted]
  ( 10pt, \ADDY + 2* \Y pt) -- ( 10pt,2.75*\ADDY+ 2* \Y pt); %% R
  \draw[->,densely dashdotted]
  ( -10pt, \ADDY + 2 *\Y pt) -- (-10pt,2.75*\ADDY+ 2* \Y pt); %% E
  \draw[->,densely dashdotted]
  ( -30pt, \ADDY + 2 *\Y pt) -- (-30pt,2.75*\ADDY+ 2*\Y pt); %% W
  \draw[->,densely dashdotted]
  ( -50pt, \ADDY + 2* \Y pt) -- (-50pt,2.75*\ADDY+ 2*\Y pt); %% Q

  %% node
  \draw[fill=black] (0pt,0pt) circle (1pt); %% rooot
  \draw[fill=white] ( 50pt, \Y pt) circle (1pt); %% Y
  \draw[fill=white] (-70pt, \Y pt) circle (1pt); %% 0
  \draw[fill=white] ( 30 pt, 2 * \Y pt) circle (1pt); %% T
  \draw[fill=white] ( 10 pt, 2 * \Y pt) circle (1pt); %% R
  \draw[fill=white] (-10 pt, 2 * \Y pt) circle (1pt); %% E
  \draw[fill=white] (-30 pt, 2 * \Y pt) circle (1pt); %% W
  \draw[fill=white] (-50 pt, 2 * \Y pt) circle (1pt); %% Q
  \draw[fill=black] ( 70pt, \Y pt) circle (1pt);


  %% elements
  \draw[fill=white] ( 50pt, -4 + \ADDY + \Y pt)
  node{\textbf{Y}} +(-4pt,-4pt) rectangle +(4pt,4pt) ; %% Y
  \draw[fill=white] ( 30pt, -4 + \ADDY +  2 *\Y pt)
  node{\textbf{T}} +(-4pt,-4pt) rectangle +(4pt,4pt) ; %% T
  \draw[fill=white] ( 10pt, -4 + \ADDY +  2* \Y pt)
  node{\textbf{R}} +(-4pt,-4pt) rectangle +(4pt,4pt) ; %% R
  \draw[fill=white] (-10pt, -4 + \ADDY + 2 *\Y pt)
  node{\textbf{E}} +(-4pt,-4pt) rectangle +(4pt,4pt) ; %% E
  \draw[fill=white] (-30pt, -4 + \ADDY + 2 * \Y pt)
  node{\textbf{W}} +(-4pt,-4pt) rectangle +(4pt,4pt) ; %% W
  \draw[fill=white] (-50pt, -4 + \ADDY + 2 *\Y pt)
  node{\textbf{Q}} +(-4pt,-4pt) rectangle +(4pt,4pt) ; %% Q

  %% desambiguator
  \draw[fill=gray!20]( 50pt, -2.5 + 2.75 * \ADDY + \Y pt)
  +(-2.5pt,-2.5pt) rectangle +(2.5pt,2.5pt);
  \draw[fill=gray!20]( 30pt, -2.5 + 2.75 * \ADDY +2 *\Y pt)
  +(-2.5pt,-2.5pt) rectangle +(2.5pt,2.5pt);
  \draw[fill=gray!20]( 10pt, -2.5 + 2.75 * \ADDY +2*\Y pt)
  +(-2.5pt,-2.5pt) rectangle +(2.5pt,2.5pt);
  \draw[fill=gray!20](-10pt, -2.5 + 2.75 * \ADDY +2*\Y pt )
  +(-2.5pt,-2.5pt) rectangle +(2.5pt,2.5pt);
  \draw[fill=gray!20](-30pt, -2.5 + 2.75 * \ADDY +2*\Y pt)
  +(-2.5pt,-2.5pt) rectangle +(2.5pt,2.5pt);
  \draw[fill=gray!20](-50pt, -2.5 + 2.75 * \ADDY +2*\Y pt) 
  +(-2.5pt,-2.5pt) rectangle +(2.5pt,2.5pt);

\end{scope}
\end{scope}

\end{tikzpicture}

%   \caption{\label{fig:lseqexample}\LSEQ data structure representing QWERTY.}
% \end{figure}

% When an insert operation is performed at a specific index in the document
% through the graphical user interface, the event is forwarded to the distributed
% data structure which translates it into \emph{allocate an identifier between the
%   identifiers adjacent to the targeted position}.  Figure~\ref{fig:lseqexample}
% shows the data structure resulting of the following editing sequence: insert 'Y'
% at 0, insert 'T' at 0, insert 'R' at 0, \ldots, insert 'Q' at 0. First, we
% observe that \LSEQ factorizes identifiers under an exponential tree
% representation. Indeed, as the depth of the tree grows, the number of potential
% children doubles. When the first insertion of Character 'Y' arrives, it
% allocates a path between the virtual adjacent identifiers paths [0] and
% [32]. Here it chooses the path [1] which is unfortunate since other insertions
% arrive preceding this character. Since there is no room for more paths at the
% first level between the virtual beginning boundary and the path of Character Y,
% the depth of the tree must grow. The new identifier is allocated between [0.0]
% and [0.64] which result in the path [0.57], etc. At the end, all identifiers are
% comparable and the resulting sequence of characters is 'QWERTY'.

An editor sends each pair of identifier and character to all members of the
editing session. For this purpose, \CRATE uses a very simple broadcast protocol
built on top of \SPRAY following the principles of epidemic dissemination (also
known as gossiping)~\cite{demers1987epidemic}. Such mechanisms make extensive
use of partial views to efficiently disseminate messages.

% Compared to Google Docs, \CRATE is decentralized and preserves
% privacy. It also supports any number of participants where Google Docs
% allows only the first fifty users to edit in real-time. Additional
% users have their rights limited to document reading.

% \subsection{Epidemic dissemination}
% \label{subsec:gossiping}

% Epidemic dissemination~\cite{birman1999bimodal,demers1987epidemic} (or
% gossiping) relies on the membership protocol to propagate messages in a scalable
% way to all members (broadcast).

\begin{algorithm}[h]
  
\small
\algrenewcommand{\algorithmiccomment}[1]{\hskip2em$\rhd$ #1}

\newcommand{\comment}[1]{$\rhd$ #1}

\newcommand{\LINEFOR}[2]{%
  \algorithmicfor\ {#1}\ \algorithmicdo\ {#2} %
  }

\newcommand{\LINEIFTHEN}[2]{%
  \algorithmicif\ {#1}\ \algorithmicthen\ {#2} %
  }

\newcommand{\INDSTATE}[1][1]{\State\hspace{\algorithmicindent}}

\begin{algorithmic}[1]
  \Function{broadcast}{$m$} \hfill \comment{$m: message$}
    \State \LINEFOR{\textbf{each} $\langle q,\,\_\, \rangle \in\mathcal{P}$}
    {$sendTo(q,\, 'broadcast',\, m)$;}
    \EndFunction
    \Statex
  \Function{receiveBroadcast}{$m$} \hfill \comment{$m: message$}
  \If {$\neg alreadyReceived(m)$}
  \State $deliver(m)$;
  \State $broadcast(m)$;
  \EndIf
  \EndFunction
  
\end{algorithmic}

  \caption{\label{algo:gossiping}Epidemic dissemination protocol.}
\end{algorithm}

Algorithm~\ref{algo:gossiping} shows the instructions of this protocol. When a
peer emits a message, it sends it to its neighborhood. Each peer receiving such
messages for the first time forwards it to its neighborhood too. Messages
transitively reach all network members.

Since the gossiping algorithm depends of the neighborhood provided by \SPRAY,
and since the latter grows logarithmically compared to the network size, the
communication complexity of an application is upper-bounded by
$\mathcal{O}(m \ln |V^t|)$, where $m$ is the space complexity of a
message. Since \CRATE's identifiers are sublinearly upper-bounded compared to
the document size, \CRATE scales well.

%%% Local Variables:
%%% mode: latex
%%% TeX-master: "../paper"
%%% End:
