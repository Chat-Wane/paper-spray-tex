
%\section{Use case: CRATE}
\section{Use case: real-time editing}

%%\subsection{Decentralized collaborative editor}
%%\label{subsec:editor}

Google Docs made real-time editing in browsers easy for millions of
users. However, Google mediates real-time editing sessions with central servers
raising issues on privacy, censorship, economic intelligence. It also raises
scalability issues in term of number of participants.  Despite that small groups
currently constitute the main range of users, events such as massive online
lectures, TV shows, conferences gather larger groups.  Google Docs supports
large groups but only the first fifty users can edit, next users have their
rights limited to document reading. In addition, real-time editing sessions can
be highly dynamic; even if only few authors are editing a document
simultaneously, the editing session includes a much larger group of potential
writers. We think that real-time editors should allow editing at anytime and
anywhere, whatever the number of participants, without third party.

Decentralized real-time editors~\cite{oster2006data, sun1998operational,
  sun2009contextbased} do not require intermediate servers and by the same solve
privacy issues. However, scalability issues remain.  Addressing scalability
requires finding a good trade-off between communication, space and time
complexities. Among others, achieving a sublinear communication complexity is
crucial for supporting large groups.  It depends on both diffusion of messages,
and the content of messages.

In this paper, we introduce \CRATE as a distributed and decentralized
CollaboRATive Editor. Thanks to \SPRAY, \CRATE builds a network of browsers
where each browsers is able to communicate with a logarithmically scaling number
of browsers compared to the global network size.  \CRATE uses a Conflict-free
Replicated Data Type~\cite{shapiro2011conflict, shapiro2011comprehensive} (CRDT)
for sequences in order to guarantee eventually consistent documents, i.e.,
documents converge to an identical state~\cite{bailis2013eventual}. CRDTs
approaches allocates an identifier to each element of the sequence representing
the document. Using the allocation function \LSEQ~\cite{nedelec2013lseq}, \CRATE
is able to keep the space of these identifiers under a polylogarithmic bound
compared to the document size. Hence, a communication complexity in
$\mathcal{O}((\log d)^2\ln(|\mathcal{N}|))$ where $d$ is the document size.

\CRATE's architecture comprises four layers:
\begin{inparaenum}[(i)]
\item communication: includes the editing session membership mechanism and the
  information dissemination protocols.
\item causality: includes the causality tracking
  structure~\cite{malkhi2007concise} that guarantees that the delivery of the
  removal of an element never precedes its insertion.
\item sequence structure: includes the structure that guarantees a global
  total order among elements of the sequence.
\item graphical user interface: includes the editor as a graphical entity that
  users can interact with inside web browsers.
\end{inparaenum}
%%The left part of the figure depicts the common process chain:
The process chain is the following: when users perform an operation on the
document, the operation is applied to the shared sequence which creates an \LSEQ
identifier. Then it decorates the result of the operation with causality
tracking metadata. Finally, \CRATE broadcasts it using the neighborhood provided
by \SPRAY.  Conversely, when \CRATE receives a broadcast message, it checks if
the operation is causally ready to be delivered. Once the condition is verified,
it applies the operation to the shared sequence which notifies the graphical
user interface of the changes.

The rest of this section details the main components of \CRATE and their
respective contributions to complexities.

\subsection{Scalable data structure for sequences}

\subsection{Epidemic dissemination}
\label{subsec:gossiping}

Epidemic dissemination~\cite{birman1999bimodal,demers1987epidemic} (or
gossiping) relies on the membership protocol to propagate messages in a scalable
way to all members (broadcast). Algorithm~\ref{algo:gossiping} shows the simple
functionning of this protocol. When a peer emits a message, it sends it to its
neighborhood. Each peer receiving such messages forwards it to its neighborhood
too. Messages transitively reach all peers of the network.

\begin{algorithm}[h]
  
\small
\algrenewcommand{\algorithmiccomment}[1]{\hskip2em$\rhd$ #1}

\newcommand{\comment}[1]{$\rhd$ #1}

\newcommand{\LINEFOR}[2]{%
  \algorithmicfor\ {#1}\ \algorithmicdo\ {#2} %
  }

\newcommand{\LINEIFTHEN}[2]{%
  \algorithmicif\ {#1}\ \algorithmicthen\ {#2} %
  }

\newcommand{\INDSTATE}[1][1]{\State\hspace{\algorithmicindent}}

\begin{algorithmic}[1]
  \Function{broadcast}{$m$} \hfill \comment{$m: message$}
    \State \LINEFOR{\textbf{each} $\langle q,\,\_\, \rangle \in\mathcal{P}$}
    {$sendTo(q,\, 'broadcast',\, m)$;}
    \EndFunction
    \Statex
  \Function{receiveBroadcast}{$m$} \hfill \comment{$m: message$}
  \If {$\neg alreadyReceived(m)$}
  \State $deliver(m)$;
  \State $broadcast(m)$;
  \EndIf
  \EndFunction
  
\end{algorithmic}

  \caption{\label{algo:gossiping}Epidemic dissemination protocol.}
\end{algorithm}

Since the gossiping algorithm depends of the neighborhood provided by \SPRAY,
and since the latter grows logarithmically compared to the network size, the
communication complexity of an application is upper-bounded by
$\mathcal{O}(m \ln |\mathcal{N}|)$, where $m$ is the space complexity of a
message. \TODO{To scale, applications are in charge of keeping $m$ under a sublinear
upper bound}.

