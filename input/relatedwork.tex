
\section{Related work}
\label{sec:relatedwork}

Part of the scalable membership protocols use the
broadcast~\cite{eugster2003lightweight, tolgyeski2009adaptive} to advertise the
existence of the members in the network. The messages contain members' identity
and timestamps. Nevertheless, the peers maintain a fixed-size partial view to
remove old entries as fresh ones arrive. This mechanism allows fast discarding
of the left/departed members but the graph converge to a topology depending of
the advertising frequency of each peers. On the other side, \SCAMPLON{} is
adaptive. Thus the partial views grow and shrink reflecting the network
size. Furthermore, the graph converges to a random topology over exchanges, and
only the convergence speed is impacted by different frequencies of the periodic
protocol. Among scalable network membership protocols,
\CYCLON{}~\cite{voulgaris2005cyclon} and \SCAMP{}~\cite{ganesh2003peer}
(cf. Section~\ref{sec:background}) are closer to \SCAMPLON{}. In \CYCLON{}, the
peers have a fixed-size partial view. Thus, the developers must foresee the
maximum size of the network implied by their application. Then, they can
configure the partial view size, the size of messages in each exchanges, and
the frequency of the exchanges. Except for the latter, these decisions cannot
be safely retracted.  \SCAMPLON{} alleviates these issues since the only
configuration concerns the exchanging frequency. Like \SCAMP{}, the partial
view size of \SCAMPLON{} adapts to the network membership. However, while the
complexity of connections depends of the size of the network in the \SCAMP{}
case, \SCAMPLON{} only use constant time connections. Therefore, the latter is
more robust to message failures at the price of a slower convergence to a
random graph topology.

To foresee the size of the partial view, one could use a network size
estimator. There exists a plethora of network size estimator, either using
\begin{inparaenum}[(i)]
\item sampling techniques~\cite{mane05network, ganesh2007peer,
    kostoulas2007active} which analyze a network subset and deduce the network
  size using probabilistic functions,
\item sketching techniques~\cite{flajolet2008hyperloglog, baquero2012extrema}
  which use hashing to compress the high amount of data and deduce the network
  size using the collisions,
\item averaging techniques~\cite{jelasity2004epidemic, blasa2011symmetric}
  which use aggregations that converge over exchanges to a value which depends
  of the network size.
\end{inparaenum}
The drawbacks of these approaches lie in their cost (e.g. communication
overhead) and/or their assumptions (e.g. a static network). On the other hand,
\SCAMPLON{} provides partial views logarithmically growing compared to the
network size. Thus, it does not imply any overhead since it converges to a
random graph by design. Conversely, \SCAMPLON{} cannot provide a precise
estimation of the network size.

Currently, the ecosystem of distributed browser-embedded applications is
blooming~\cite{firefoxhello, sharefest, webtorrent}. Indeed, the arrival of
peer-to-peer within modern browsers opened the gate to a large variety of
real-time distributed web applications, previously quartered to standalone
applications. As consequence, they become accessible to a larger audience. A
majority of these applications targets a small number of simultaneous users
(e.g. Video conferencing~\cite{firefoxhello}) but more recently, applications
aiming larger scale have been arising. For instance,
WebTorrent~\cite{webtorrent} is a file-sharing system based on the BitTorrent
protocol. Nevertheless, the users of this application share static data. As
such, the built network could be completely clustered but still working as long
as one member in each cluster has all the data.  Till now, the class of
applications requiring scalable broadcast has been neglected. The goal of
\SCAMPLON{} is to fill this gap by providing a scalable membership protocol
well-adapted to such context. Using \SCAMPLON{}, the network is connected, yet
using few connections. Therefore, the broadcast messages reach all members very
efficiently.


%%% Local Variables:
%%% mode: latex
%%% TeX-master: "../paper"
%%% End:
