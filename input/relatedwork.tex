
\section{Related work}
\label{sec:relatedwork}

\subsection{Distributed ecosystem within browser}
Currently, the ecosystem of distributed browser-embedded applications is
blooming. Indeed, the arrival of peer-to-peer within modern browsers opened the
gate to a large variety of real-time distributed web applications, previously
quartered to standalone applications. As consequence, they become accessible to
a larger audience. A majority of these applications targets a small number of
simultaneous users (e.g. Sharefest~\cite{sharefest}). Therefore, they can
afford the use of fully connected graph (e.g. PeerJS~\cite{peerjs}). More
recently, applications aiming larger scale have been arising. For instance,
WebTorrent~\cite{webtorrent} is a file-sharing system based on the BitTorrent
protocol. Nevertheless, users of this application share static data. As such,
the built network could be completely clustered but still working as long as
one member in each cluster has all the data.  Till now, the class of
applications requiring scalable broadcast has been neglected. The goal of
\SCAMPLON{} is to fill this gap by providing a scalable membership protocol
well-adapted to such context. Using \SCAMPLON{}, the network is connected, yet
using few connections. Therefore, the broadcast messages reach all members
either directly, or transitively.


\subsection{Network size estimators}
There exists a plethora of network size
estimator~\cite{kostoulas2007active,baquero2012extrema}, either using
\begin{inparaenum}[(i)]
\item sampling technics~\cite{ganesh2007peer,kostoulas2007active} which analyse
  a subset of the network and deduce the network size using probabilistic
  functions,
\item sketching technics~\cite{baquero2012extrema} which use hashing to
  compress the high amount of data and deduce the network size using the
  collisions,
\item averaging technics~\cite{jelasity2004epidemic} which use aggregations
  that converge over exchanges to the network size.
\end{inparaenum}
Using such estimator to resize the partial view of each peer is
possible. However, these approaches are often very costly, and can assume some
network properties (e.g. no dynamic network during an estimating). On the other
hand, \SCAMPLON{} provides partial views logarithmically growing compared to
the network size. Thus, it implies no overhead since it converges by design to
a random graph. Then, each peer is able to compute a rough estimate of the
network size.

\subsection{Network topologies}

%%% Local Variables:
%%% mode: latex
%%% TeX-master: "../paper"
%%% End:
