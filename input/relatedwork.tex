
\section{Related work}
\label{sec:relatedwork}

Currently, the ecosystem of distributed browser-embedded applications is
blooming~\cite{firefoxhello,sharefest,webtorrent}. Indeed, the arrival of
peer-to-peer within modern browsers opened the gate to a large variety of
real-time distributed web applications, previously quartered to standalone
applications. As consequence, they become accessible to a larger audience. A
majority of these applications targets a small number of simultaneous users
(e.g. Video conferencing~\cite{firefoxhello}) but more recently, applications
aiming larger scale have been arising. For instance,
WebTorrent~\cite{webtorrent} is a file-sharing system based on the BitTorrent
protocol. Nevertheless, users of this application share static data. As such,
the built network could be completely clustered but still working as long as
one member in each cluster has all the data.  Till now, the class of
applications requiring scalable broadcast has been neglected. The goal of
\SCAMPLON{} is to fill this gap by providing a scalable membership protocol
well-adapted to such context. Using \SCAMPLON{}, the network is connected, yet
using few connections. Therefore, the broadcast messages reach all members very
efficiently.

To foresee the size of the partial view, one could use a network size
estimator. There exists a plethora of network size
estimator~\cite{jelasity2004epidemic, ganesh2007peer, kostoulas2007active,
  baquero2012extrema}, either using
\begin{inparaenum}[(i)]
\item sampling techniques~\cite{mane05network, ganesh2007peer,
    kostoulas2007active} which analyse a network subset and deduce the network
  size using probabilistic functions,
\item sketching techniques~\cite{baquero2012extrema} which use hashing to
  compress the high amount of data and deduce the network size using the
  collisions,
\item averaging techniques~\cite{jelasity2004epidemic} which use aggregations
  that converge over exchanges to a value which depends of the network size.
\end{inparaenum}
The drawbacks of these approaches lies in their cost (e.g. communication
overhead) and/or their assumptions (e.g. a static network). On the other hand,
\SCAMPLON{} provides partial views logarithmically growing compared to the
network size. Thus, it does not imply any overhead since it converges to a
random graph by design. Conversely, \SCAMPLON{} cannot provide a precise
estimation of the network size.

%%% Local Variables:
%%% mode: latex
%%% TeX-master: "../paper"
%%% End:
